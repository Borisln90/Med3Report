\section{Discussion}
Throughout this discussion the project will be reflected upon, as a whole. This includes looking at the choice of alternative structure for the project, as to see what occurred in this new form compared to the generally used structure. Additionally, there will be a discussion on what was achieved through the different steps of the project, e.g. the research, implementation and evaluation phases. All this discussion will ultimately lead to some pointers to tell how well the problem, formulated for this project, was solved, i.e. to what extends the problem was solved and what this can be used for.
\bigskip

The structure of the entire report was to emphasize suggestions of designs during development of, not only the product, but also the process of obtaining information and knowledge to form the final problem statements and the requirements. This method allowed the possibility to create several design templates that would plausibly suffice as a solution to the specified problem statement. Based upon the test results, there are no real indications towards the chosen design being the most efficient solution out of several suggestions. Although, by evaluating the test results, as to find the elements that were dysfunctional in any matter, it was possible to create a redesign of the implemented product, to plausibly suffice as the most efficient solution based upon the acquired research and test results.
\bigskip

The parameters of the list of requirements were clear; i.e. the design would have to be completed within the boundaries of, among other things, a standard webcam and e.g. ordinary household remedies such as cardboard and colored paper. The list of requirements is developed to be seen as parameters for which, if implemented, is successful to answer the final problem statement. There is, however, several methods for which each requirement can be implemented, and does therefore require speculations as to how the actual implementation has been constructed, and to what degree the success of said requirement is fulfilling its purpose of answering to the final problem statement. The requirements, describing general hardware and software does not count towards requirements which are subject for discussion, since the sheer development of the artefact is fulfilling the specified requirements e.g. that the controller must require a standard webcam.
\bigskip

The remaining requirements are, however, interpretations and conducted choices as to how each specific requirement are implemented. By interpreting the data, gained from observations and interviews, it was possible to interpret and deepen the understanding of how each requirement could have been conducted differently. The functionalities were e.g. defined as movement and additional functions such as gear shift and alternative camera angles. The outcome of the preliminary test indicated that our constructed solution to the gear-shift function was inefficient, therefore it was never tested, but interview answers indicates that it would have been a prohibitive feat to manage while steering the vehicle.
\bigskip

In accordance with the final problem statement the initial purpose of the project was to establish whether or not a controller could be created that would be comparable to an undefined SOTA-controller in terms of functionalities.
Through the strategy of the laid out test-methodology it was, among other things, recorded that the test subjects generally did comprehensively better with the SOTA-controller i.e. completed lap-time. This indicates that the comparison is imbalanced, meaning that a play through with the SOTA-controller is more effective i.e. the functionalities are advantageously implemented, compared to the created artefact. Such an outcome is expected from a general comparison between the two products; however, the intention of the product was not to be on par with a SOTA-solution, but merely to create a comparative product from relatively simple remedies. In this respect, it is assessed that the problem statement has been answered to a partial degree.	
\bigskip

Going through the process of this project and stepping towards an implemented attempt of a solution to the defined problem, formulated in the initial problem statement, and further specified in the final problem statement, several interesting areas and aspects of visual computing were revealed. Starting with the research gathered, to clarify the problem statement, several new ways of analysing, and processing, a video feed, previously completely/partially unknown to us, gave a much better understanding of what visual computing is, in its raw sense. Additionally the research opened up our perspective on what might actually be possible, when trying to reach for a webcam-only-dependent game controller that would be comparable with the more modern technological gaming platforms/controllers such as the State of the Art (SOTA) products i.e. in this case, an Xbox controller.
\bigskip

The acquired research found, led to designing of a plausible solution to the problem, different designs were developed, in order to widen the perspective of the possibilities to solve said problem statement. These designs all had a part to themselves, which made them unique, yet still all relevant as to how the functionality, required by the games, could be implemented. Even though they all had their differences, it still seemed like all of them would be able to utilize almost the exact same methods, in regards to visual computing (Image Processing and -Analysis), showing how many possibilities there are in visual computing. However, through the implementation it became clear that some of the methods were much more demanding in relation to time consumption, and required knowledge. This ultimately lead to an implementation phase, where some difficult choices had to be made regarding whether the prototype should be able to obtain user input easily and quickly, with the downside of being imprecise and dependant on environmental lighting.
\bigskip

Additionally troubling thoughts, from before the actual research began, was whether a final prototype could be made possible to use with an already existing game on the market, or if it would require a development of a new game. Moving towards the implementation of the prototype different ways of solving this problem were found, yet all of them seemed to rely on the same kind of communication between the software developed for the prototype and the other application, i.e. the racing game. As earlier described in the usability test \ref{sec:usability}, this communication did seem to suffer from some flaws, which, at that moment, showed to be a big obstacle when trying to make the game controller able to handle several different functions at a time. This resulted in a prototype with the disability of not being able to handle the user’s input for gear shifting, together with the other specified functions. Although it seemed that the test and evaluation of the prototype would be suffering from this disability, it still yielded positive data. This data has the potential of giving directions for what a further development of this prototype would need, in order to accommodate the requirements and functionalities revealed through the research.
\bigskip

During the implementation phase, this proved to be true, although some actions were difficult to implement - theories as to why these problems occurred, is explained in the Evaluation chapter, analysis results \ref{sec:analysisresults}, and will not be repeated here. The product-test brought some new issues to light; our testers had trouble using the controller, even with the removal of the gear shift function. 

Through the strategy of the laid out test-methodology it was among other things recorded that the test subjects generally did comprehensively better with the SOTA-controller i.e. completed lap-time. This indicates that the comparison is imbalanced, meaning that a play through with the SOTA-controller is more effective i.e. the functionalities are advantageously implemented compared to the created artefact.
Such an outcome is expected from a general comparison between the two products; however, the intention of the product was not to be on par with a SOTA-solution, but merely to create a comparative product from relatively simple remedies. In this respect, it is assessed that the problem statement has been answered to a partial degree.
\bigskip

“Can a vehicle game controller be created, using a standard web cam that utilizes visual computing with functionalities, that are comparable to state of the art game controllers in terms of functionality?”
\bigskip

The question of the final problem statement seeks to answer whether a game controller can be created using a webcam instead of specialized hardware. During the testing it was found that the general usage of the developed product was functional, but with general software issues which could have been modified, added or deleted as proposed in the re design section. 
Based upon the test and the general interpretation of the results it can be concluded that the game controller indeed can be created, using a standard webcam utilizing visual computing. However, the functionalities are interpretations of implemented functionalities of which has been implemented in a manner which was deemed as the best plausible design solution. These comparable functionalities are however, when evaluating the test results and the redesign, subjected to change to further improve the constructed usage of hardware and the developed software.
