\subsection{Game controls in various racing games}
\label{gamecontrols}
Within the State Of The Art of the formulated (IPS) area, we have to establish which choices are popular when it comes to deciding how you should be able to control your racing car in the game. This is needed in order to determine which requirements there are to the design of the controller, so that the user will be able to do the same actions in the game, as he/she would be able to perform in the same game played using e.g. Wii or Kinect.

By looking into which games in the racing category are popular on the three gaming platforms Nintendo Wii, Xbox Kinect and PlayStation Move, we can quickly see a difference in the variety of racing games for the different platforms. It seems that finding a racing game for PlayStation Move can be somewhat troublesome, and the most prominent racing game designed for use with PS Move seems to be the game: \textbf{LittleBigPlanet™ Karting} \parencite{Miller2012}. Moving on to the Xbox Kinect; here it also seems that there is only one protruding game  when looking for the most popular racing game designed for the Kinect without the use of additional controllers, i.e. \textbf{Kinect Joy Ride}\parencite{Davidson2010}. Nintendo Wii, on the other hand, seems to be the most popular gaming platform of the three if you want to play a car racing game. For the Wii there are several different racing games that seem to be popular. Amongst those, we have various versions of \textbf{Need for Speed}, \textbf{TrackMania: Build To Race} and \textbf{Mario Kart Wii}, and \textbf{Sonic \& SEGA All-Stars Racing}, which looks much like Mario Kart in the controls. (This will be explained below) \parencite{Ign2013}.
In order to establish a general rule to follow when you want to create a controller that should be usable with most of the racing games available, a list of the different controls used by the games needs to be made. Making the foundation using the aforementioned 5 games we can assemble the following list:
\bigskip

\noindent Consistent for all five games:
\begin{itemize}
\item Steering left and right
\item Acceleration
\item Braking / Reversing
\item Changing camera view (at least for the Wii games - different views from game to game)
\item Display a pause menu
\end{itemize}

\noindent Need For Speed (Wii):
\begin{itemize}
\item Nitro (speed boost)
\item Change gear up and down
\end{itemize}

\noindent TrackMania (Wii):
\begin{itemize}
\item Braking while accelerating (Drifting)
\end{itemize}

\noindent Mario Kart Wii:
\begin{itemize}
\item Choose item
\item Throw items back and forward
\item Perform mid-air tricks
\end{itemize}

\noindent LittleBigPlanet™ Karting (Move):
\begin{itemize}
\item Use weapons
\end{itemize}

\noindent Kinect Joy Ride (Kinect):
\begin{itemize}
\item Charge and release speed boost
\item Drift to either side
\item Use item
\item Perform mid-air tricks in different directions
\end{itemize}

From this list, it can be seen that there are some general controls that are used through all of these different games. This means that when designing some form of game controller for a car racing game, you should make the user able to perform the five commands listed at the top, as those commands seem to be important, when they are consistent through multiple games. Furthermore, depending on what game we are talking about, there seems to be about 1 – 4 additional controls to be able to play the game fully.

