\subsection{Visual Computing}
Visual Computing is computing that lets the user interact with, and/or control work by manipulating visual images \parencite{Rouse2013}. The visual images can be photographs, 3D-scenes, video sequences or any other visual output for that matter.

It is the core of controlling and manipulating images and is therefore an important aspect to include in the research and find already established solutions to this method.
\bigskip

One example of Visual Computing is Camspace\copyright. Camspace, is a platform for visual computing that through the use of a standard webcam detects object and hand gestures and then utilize those to be implemented in games, as a mouse controller etc. \parencite{Camspace2009}

It works as a software platform that is installed on any computer with access to a basic webcam, where options to determine the use dictates how the user interacts with it through human gestures with or without an object.
\bigskip

The program can utilize an object as a controller by presenting a solid object with a uniform color that does not match the environment i.e. the shirt that is worn or the wall behind the player. Also, the room must be lit for the color recognition to be possible.
\bigskip

Camspace do offer games and programs that are specifically designed for the software itself, but does also contain drivers for it to be compatible with preexisting games, which is developed with other means of control. Although the games are compatible with the software, it is only a limited amount of controls that can be implemented to function with the program.

\begin{table}[h]
\begin{tabular}{| l | l }
Game Title & In-game functionalities\\
\hline \hline
Need for speed underground 2 & Move left/right \& accelerate/break\\
Aquadelic & Left/right \& horizontal\\
Trackmania ( \& Nations) & Speed(depth of object) \& left/right\\
LudoRace & Left/right\\
OffRoadArena & Speed(depth of object) \& left/right\\
Flight Model Simulator & N/A\\
Need for speed Carbon & N/A\\
\end{tabular}
\caption{List of vehicle based (driving, sailing, flying) games and the in-game functionalities which is utilized by the Camspace software\parencite{Camspace2009}}
\label{tab:camspace}
\end{table}

\subsubsection*{Summary}
The research indicates that it is possible to create software for computers that utilizes simple webcams for visual computing purposes, such as game control and interaction. It is also possible to modify the use of the program in order to access and control games that is not specifically developed for Camspace itself. Although it is possible, it is only a limited amount of controls that can be imported and utilized via Camspace; (See table \ref{tab:camspace})  such as speed increase/decrease by pulling the designated control object closer or further away, and turn/rotation by rotating the object left or right.