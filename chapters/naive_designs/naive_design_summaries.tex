\subsection{Naïve design summaries}
The naïve design ideas are presented in a short and precise manner for summary purposes. Only essential information and documentations are described. For a detailed description and explanation for each specific naïve design read appendix \ref{NaiveDesigns}. 

\subsubsection{Control a car through gestures - Appendix \ref{nd1}}
 The idea is to make a vehicle controller, out of a camera, that can react to a user’s hand gestures and translate that into input for a vehicle based application or game. These gestures would translate to tasks commonly used in relation to vehicle games, like acceleration,braking , change gears and hand breaking. The system should also be used to control the vehicles directions.

\subsubsection{Controller Blocks - Appendix \ref{nd2}}
The idea is to create a vehicle controller that utilizes blocks which the players move around on a flat surface in order to navigate, along with additional functionalities, the specified vehicle games. This idea supports both single and multiplayer, for the basic standard webcam will be pointed down on the flat surface and will be able to track two players navigating their blocks separately.

\subsubsection{Immersive Racing Game Experience - Appendix \ref{nd3}}
The idea is to create a controller that uses a standard basic webcam to track the whole body of a person, allowing movement, gestures and face recognition to control the presented game through defined gestures and movements.

\subsubsection{Pointing Controllers - Appendix \ref{nd4}}
The idea behind this concept is to use a webcam to track finger(s) on a plain surface to register movement and functionalities of the designated game. By relocating the finger(s), depending on the game, the input will act accordingly to the specific game/functionality.

\subsubsection{Object Controlled Visual Processing - Appendix \ref{nd5}}
The idea revolves around a multicolored - or shape-recognizable object that can be moved around in front of the webcam to create a motion capture-device that can be used as a logical vehicle controller e.g. a plate as seen in the image for controlling a car or boat etc.

\subsubsection{Race-Making game - Appendix \ref{nd6}}
The idea is to have a controller utilizing a board where you place down the bricks as “points” where you would like the track to go. The game then connects these points with lines, and there you have a track to maneuver the vehicle inside the specified vehicle game. To control the vehicle, a user’s finger is traced through a static webcam.

\subsubsection{Tabletize your computer - Appendix \ref{nd7}}
The idea for this design is, using a standard webcam, simulating the rotation measuring devices of modern tablets and smart phones, allowing computer users the same intuitive means of controlling games.

As the webcam is recording, edges will be recognized using image processing, and the movement of said edges will be converted into movement on the computer - In other words, movement within the captured pictures will be replacing the tilting sensors of the tablets and smart phones.

