\subsection{Focusing the research areas}
Looking through the naïve designs (see appendix), it was noticed that most of the ideas had common research areas. So we took the best of the ideas and had a look at the requirements for each naïve idea. A list of required research areas became apparent from the studies of the naïve designs.

The list has been prioritized according to importance for the project - determined by a combination of how many of the designs listing a specific requirement and by how a general a requirement is (i.e. can it be used for a design where it is not listed initially).

The reason for prioritizing the list in such a way, is to keep it short - filling the report with research, not being used is inefficient in terms of time used and can be disruptive for the reader - because of this, "unimportant" research have been excluded (or in some cases, put into the appendix).
In the appendix, a table of how each topic have been evaluated and prioritized.
\bigskip

The order of importance of the research areas is listed here and described below:
\begin{itemize}
\item SOTA (State Of The Art)
\item Image Processing and Analysis (IP \& IA)
\item Transferring data from one application to another
\item Gesture recognition
\item Controller appeal / motivation of the users
\item Networking
\item Webcam differences
\end{itemize}

\subsubsection{SOTA (State Of The Art)}
To compare a new game controller with something already in production, the first step is to do research on the existing solutions. More specifically, research should revolve around how they work, why these products are popular, and try to find out how it was designed and developed, to avoid repeating their mistakes. This research can be found in the analysis (chapter: \ref{sec:analysis}).

\subsubsection{Image Processing and Analysis (IP \& IA)}
As this project is going to be heavily based on video / image processing and analysis, it is very important to research how to use the different techniques and methods associated with working in these fields. As can be seen in appendix \ref{NaiveDesigns}, most of the naïve designs focus on using color recognition, pattern recognition and edge detection - making these techniques especially important to research. Additionally, as some of the designs include the possibility to allow multiple players to play at the same time, blob detection is also important to research, as this method allows the program to distinguish between different elements in an image.

\subsubsection{Transferring data from one application to another}
As the focus of this project is not creating a game, but rather creating a controller for a game, a key component of designing the controller, is enabling it to transfer data from the image-capture into a 3rd party application (i.e. a game). Whether the control will transfer direct data (i.e. creates a new input for the game) or simply "mimics" existing game controls (e.g. a specific movement is turned into a specific button push on a keyboard).

To do this, some research of how to make two applications communicate accurately and efficiently is required.

\subsubsection{Gesture recognition}
As gesture recognition is used for some of the naïve designs, this field should also be researched. While this specific method of image processing is not as important as the other, as many of the features of gesture recognition could possibly be done, using pattern recognition instead, it is still a relevant topic for this report, and as such, should be researched.

\subsubsection{Controller appeal / motivation of the users}
In order to improve the general user experience, some research into controller appeal and how to motivate a user should be made including - but not limited to - physical design, user interface and user controls (which gestures are most appropriate to use etc.).

\subsubsection{Webcam differences}
The idea with this project, as mentioned, is to make a solution that is cheap for the users, which means that, as most people already has a webcam, a webcam would not be required for the user to buy to use most of the naïve design ideas. As people might have very different webcams some webcams might require a different setup in order to be able to use the product. To be able to create a product that can be used by all kinds of webcams it is therefore required to know what distinguishes the different webcams from one another, such as how good they are at grabbing the true colors etc.