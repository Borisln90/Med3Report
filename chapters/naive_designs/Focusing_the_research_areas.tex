\subsection{Focusing the research areas}
Looking through the naïve designs (Appendix \ref{NaiveDesigns}), it was noticed that most of the ideas had some common research areas. A list of required research areas became apparent from the studies of the naïve designs.

The list has been prioritized according to its importance for the project, determined by a combination of how many of the designs listed a specific requirement and in terms of importance, i.e. can it be used for a design where it was not listed initially.

The reason for prioritizing the list in such a way, is to keep it short and relevant. Filling the report with redundant research is inefficient in terms of time used and can be disruptive for the process of the whole project.
\bigskip

The order of importance of the research areas is listed here and described below:
\begin{itemize}
\item State Of The Art(SOTA)
\item Image Processing and Analysis (IP \& IA)
\item Transferring data from one application to another
\item Gesture recognition
\item Controller appeal / motivation of the users
\item Networking
\item Webcam differences
\end{itemize}

\subsubsection{State Of The Art(SOTA)}
To compare a new game controller with something already in production, the first step is to do research on the existing solutions. More specifically, research should revolve around how they work, why these products are popular, and find out how it was designed and developed, to avoid falling into the same pit. Starting out with some research in this area will also give a better understanding of what is needed to solve the IPS.

\subsubsection{Image Processing and Analysis (IP \& IA)}
As this project is going to be heavily based on video / image processing and analysis, because a webcam should be used, it is very important to research how to use the different techniques and methods associated with working in these fields. As shown in appendix \ref{NaiveDesigns}, most of the naïve designs focus on using colour recognition, pattern recognition and edge detection, making these techniques especially important to research. Additionally, as some of the designs include the possibility to allow multiple players to play at the same time, BLOB detection is also important to research.

\subsubsection{Transferring data from one application to another}
As the focus of this project is not lying on creating a game, but rather to create a controller for a game, a key component of designing the controller, is enabling it to transfer data from the camera into a 3rd party application, i.e. a game. This means whether the controller will transfer direct data, i.e. creating a new input for the game, simulate existing game controls, such as a keyboard, or a completely different way.

To do this, some research on how to make two applications communicate accurately and efficiently is required.

\subsubsection{Gesture recognition}
As gesture recognition is used for some of the naïve designs, this field should also be researched. While this specific method of image processing/analysis is not as important as the other specified research areas, many of the features of gesture recognition could possibly be done using pattern recognition instead. It is still a relevant topic for this project, and as such, should be researched.

\subsubsection{Controller appeal / motivation of the users}
In order to improve the general user experience, some research into controller appeal and how to motivate a user should be made including, but not limited to, physical design, user interface and user controls. Additionally this includes which gestures are most appropriate for a user to use in order to make it feel more natural/intuitive.

\subsubsection{Webcam differences}
Another part of this project is, as mentioned, to make a solution that is cheap for the user. Which means that people with laptops will most likely have webcams embedded into it. A webcam would then not be required for the user to buy to use most of the naïve design ideas. As people might have very different webcams, some webcams might require a different setup in order to be able to use the product because of quality of the designated webcam etc. To be able to create a product that can be used by all kinds of webcams it is required to know what distinguishes the different webcams from one another or at least find a minimum requirement for the webcam to be used.