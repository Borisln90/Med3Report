\section*{Readers guide}
\textit{Dear reader, please note that the structure of this report differs from the average Aalborg University(AAU)-group reports, and thus requires some clarification before the perusal is initiated. 
The structure has been introduced by our supervisor, as an experimental attempt to avoid the impasse that occasionally occurs in the start-up phase.
The fundamental difference between the structures is that the investigation is replaced with a concept known as Naïve Designs. Instead of building the final problem statement upon delimitations (the usual purpose of the investigation is to narrow the focus of the problem, and thus end up with a focused direction), the Naïve Designs is an alternative way of designing solutions to the defined problem statement.}
\bigskip

\textit{\textbf{The definition of a Naïve Design} is a design/concept idea without any core-technical substantiation. 
The concepts are merely developed from a brainstorming session where feasibility is subordinate, since it might cause limitations to the creativity if that is to be considered.	
The Naïve Designs will subsequently be considered in more detailed perspective, and we will critically approach and analyse the design-ideas, including their target groups, strengths, weaknesses and research focus. 
Based on the findings, one idea will be favored and the continuing analysis will relate to the prioritized design.}
\bigskip

\textit{\textbf{The Design phase} will require three different design solutions to the final problem statement, and again only one design is to be chosen for the final implementation. 
Theoretically this will encourage the creativity when two additional designs has to be drafted, and any viable ideas from the three different designs can and should be implemented into the final design.}
\bigskip

\textit{A project-report usually ends with an evaluation that reflects upon the test results, and concludes on the success, faults and failures of the design and implementation. 
Seldom these evaluation ends in recreation and reconsideration of the concept. 
In this report we will, after the evaluation, reconsider our final design idea, and based on the data gained from the test results provided, generate a new theoretical re-design.}
\bigskip

\textit{Thank you for showing interest in our research, and we hope you will enjoy our report as much as we enjoyed making it.}
\bigskip

\noindent\textit{Regards,}

\noindent\textit{Group 307}