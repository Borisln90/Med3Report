\section{Naïve designs}
The usual structure of a project like this starts with developing an idea of some problem and formulate this into an initial problem statement. This is followed by doing some initial research, called investigation, with the goal of turning the idea into an final and more specific problem statement, which is then used to narrow the field of research into an analysis, with the goal of specifying a list of requirements for a design.

This report structure here is somewhat similar in content, but different in approach:
After defining the initial problem statement it is used as a framework for creating some naïve designs. These designs are then used to figure out which areas of research should be focused on, i.e. they effectively work as the investigation chapter.
\bigskip

The reason this approach is used, is to increase efficiency and broadening the perspective of possible solutions to a problem. Instead of doing a lot of preliminary research on a subject, in order to create a final problem statement, the order is reversed, which reduces the amount of work that "goes to waste" i.e. ends up not being important for the project, and instead of risking locking upon a single solution, this approach tries to focus on many possibilities to solve a problem.


\subsection{Method: Naïve designs}
The naïve designs are, as the name implies, designs based on assumptions and ideas, instead of actual research or considerations concerning feasibility taken into account. The designs are not meant to be used as a final design, but rather as inspiration source for which areas needs further research, before finalizing the design.

The research which will be considered relevant is the areas of research which proves to be indispensable or in any way plausible to be of relevance to the project. Ultimately, finding the areas of research based on the content of the naïve designs, will help to clarify or in any way specify how to create a product.
\bigskip

The designs start as a brain storm, revolving around ways in which the initial problem statement \textit{could} be answered. The ideas from this brain storm, will then be used to create some design concepts, explaining how a product can be created, how it is intended to work and how it is used to answer the problem. Once the designs are made, thought will go into which elements needs further research, in order to actually make the product described, and to make sure that it \textit{actually} tests, what it is supposed to.
\bigskip

This approach of "Initial Problem Statement -> Naïve Design -> Analysis -> Final Problem Statement" replaces the traditional "Initial Problem Statement -> Investigation -> Final Problem Statement -> Analysis", and the reason this is done, is an attempt at avoiding the risk of getting tunnel vision of a specific way to solve a problem - Instead of focusing on a single way of solving a problem right from the get-go, this approach revolves around having multiple solutions in mind, right up until the point, where the \textit{best} (potentially, at least) solution have been found, through the process of delimitation. 

By having several solutions to the specific problem, the goal is to either choose or combine aspects of the design ideas to conduct a general tender to the solution.

\subsection{Naïve design summaries}
The naïve design ideas are presented in a short and precise manner for summary purposes. Only essential information and documentations are described here below. For a detailed description and explanation for each specific naïve design, read appendix \ref{NaiveDesigns}. 

\subsubsection{Control a car through gestures - Appendix \ref{nd1}}
 The idea is to make a vehicle controller, out of a camera, that can react to a user's hand gestures and translate that into input for a vehicle based application or game. These gestures would translate to tasks commonly used in relation to vehicle games, like acceleration, braking, change gears and hand braking. The system should also be used to control the vehicles directions.

\subsubsection{Controller Blocks - Appendix \ref{nd2}}
The idea is to create a vehicle controller that utilizes blocks which the players move around on a flat surface in order to navigate, along with additional functionalities, the specific vehicle in a game. This idea supports both single- and multiplayer. The webcam capturing the user's movement will be pointing down on the flat surface and will be able to track the players navigating their blocks separately.

\subsubsection{Immersive Racing Game Experience - Appendix \ref{nd3}}
The idea is to create a controller that uses a standard webcam to track the whole body of a person, allowing movement, gestures and face recognition to control the presented game through defined gestures and movements.

\subsubsection{Pointing Controllers - Appendix \ref{nd4}}
The idea behind this concept is to use a webcam to track movement of the user's finger(s) over a plain surface. The registered movements are used as input to activate functionalities in the designated game. By relocating the finger(s), depending on the game, the input will act accordingly to the specific game/functionality.

\subsubsection{Object Controlled Visual Processing - Appendix \ref{nd5}}
The idea revolves around a multicolored - or shape-recognizable object, such as a colored piece of paper, that can be moved around in front of the webcam to create a motion capture-device. This device can be used as a logical vehicle controller where the user would e.g. rotate to turn the vehicle or raise the controller to accelerate.

\subsubsection{Race-Making game - Appendix \ref{nd6}}
The idea is to have a controller utilizing a board where you place down some bricks as “points” to where you would like the track to go. The game then connects these points with lines, and there you have a track to maneuver the vehicle inside the specified vehicle game. To control the vehicle, a user’s finger, moving along the track, is traced through a static webcam.

\subsubsection{Tabletize your computer - Appendix \ref{nd7}}
The idea for this design to simulate the rotation measuring devices of modern tablets and smart phones, using a standard webcam, allowing computer users the same intuitive means of controlling games.

As the webcam is recording, edges will be recognized using image processing, and the movement of said edges will be converted into movement on the computer. In other words, movement within the captured images will be replacing the tilting sensors of the tablets and smart phones.


\subsection{Focusing the research areas}
Looking through the naïve designs (see appendix), it was noticed that most of the ideas had common research areas. So we took the best of the ideas and had a look at the requirements for each naïve idea. A list of required research areas became apparent from the studies of the naïve designs.

The list has been prioritized according to importance for the project - determined by a combination of how many of the designs listing a specific requirement and by how a general a requirement is (i.e. can it be used for a design where it is not listed initially).

The reason for prioritizing the list in such a way, is to keep it short - filling the report with research, not being used is inefficient in terms of time used and can be disruptive for the reader - because of this, "unimportant" research have been excluded (or in some cases, put into the appendix).
In the appendix, a table of how each topic have been evaluated and prioritized.
\bigskip

The order of importance of the research areas is listed here and described below:
\begin{itemize}
\item SOTA (State Of The Art)
\item Image Processing and Analysis (IP \& IA)
\item Transferring data from one application to another
\item Gesture recognition
\item Controller appeal / motivation of the users
\item Networking
\item Webcam differences
\end{itemize}

\subsubsection{SOTA (State Of The Art)}
To compare a new game controller with something already in production, the first step is to do research on the existing solutions. More specifically, research should revolve around how they work, why these products are popular, and try to find out how it was designed and developed, to avoid repeating their mistakes. This research can be found in the analysis (chapter: \ref{sec:analysis}).

\subsubsection{Image Processing and Analysis (IP \& IA)}
As this project is going to be heavily based on video / image processing and analysis, it is very important to research how to use the different techniques and methods associated with working in these fields. As can be seen in appendix \ref{NaiveDesigns}, most of the naïve designs focus on using color recognition, pattern recognition and edge detection - making these techniques especially important to research. Additionally, as some of the designs include the possibility to allow multiple players to play at the same time, blob detection is also important to research, as this method allows the program to distinguish between different elements in an image.

\subsubsection{Transferring data from one application to another}
As the focus of this project is not creating a game, but rather creating a controller for a game, a key component of designing the controller, is enabling it to transfer data from the image-capture into a 3rd party application (i.e. a game). Whether the control will transfer direct data (i.e. creates a new input for the game) or simply "mimics" existing game controls (e.g. a specific movement is turned into a specific button push on a keyboard).

To do this, some research of how to make two applications communicate accurately and efficiently is required.

\subsubsection{Transferring data from one application to another}
As gesture recognition is used for some of the naïve designs, this field should also be researched. While this specific method of image processing is not as important as the other, as many of the features of gesture recognition could possibly be done, using pattern recognition instead, it is still a relevant topic for this report, and as such, should be researched.

\subsubsection{Controller appeal / motivation of the users}
In order to improve the general user experience, some research into controller appeal and how to motivate a user should be made including - but not limited to - physical design, user interface and user controls (which gestures are most appropriate to use etc.).

\subsubsection{Webcam differences}
The idea with this project, as mentioned, is to make a solution that is cheap for the users, which means that, as most people already has a webcam, a webcam would not be required for the user to buy to use most of the naïve design ideas. As people might have very different webcams some webcams might require a different setup in order to be able to use the product. To be able to create a product that can be used by all kinds of webcams it is therefore required to know what distinguishes the different webcams from one another, such as how good they are at grabbing the true colors etc.