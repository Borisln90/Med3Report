\subsection{Design delimitation}

To delimit the three designs, to the actual design that will be implemented, each design will be compared to the list of requirements. Here the design ideas will be evaluated in terms of how well each specific design fulfil the requirements and therefore might answer as a possible solution to answer to the final problem statement. See section \ref{LOR} for the list of requirements

\begin{itemize}
\item The controller must be for controlling vehicle games.\newline
Each design is specifically developed with vehicle games in mind. Therefore no specific features of any of the ideas is relevant in terms of this specific requirement.

\item The controller must utilize visual computing within the limitations of a webcam.\newline
Each design idea is based on a standard webcam. No design idea has any special requirements beyond those defined in section \ref{webcam}. 

\item The controller must be comparable to pre-existing controllers in terms of functionality.\newline
All of the design ideas are developed based on the commonly utilized controls as researched in \ref{gamecontrols}. Therefore this requirement is fulfilled by each of the three design ideas.

\item The solution must require a standard webcam.\newline
All three design ideas is based on utilizing a standard webcam.

\item The solution must require a computer to run the software.\newline
All three design ideas is based on utilizing a computer to run the software.

\item the standard webcam does not have to be embedded in the computer. Therefore computer type is irrelevant.\newline
None of the design ideas specify additional requirements to the computer that should be used, therefore this requirement is fulfilled by all three design ideas.

\item No additional hardware required for the sake of control of the vehicle game.\newline
The design ideas specifies that only computer hardware is required. Additional items are accessories which must be utilized to control the software on said hardware, and does therefore not count towards additional hardware. This is the case for all three design ideas.

\item The controller software must be compatible with a pre-existing vehicle game which supports
the game functionalities as described in Controller functionalities.\newline
Each design idea utilizes profiles to map the output to each game. This means that the controller should be compatible with vehicle games that might deviate in control scheme.

\item The output of the software should mimic input to conventional input methods like keyboard
presses or mouse movements.\newline
The design ideas simulates keyboard presses to send commands to the game. 

\item the controller software must track either colors, patterns or both.\newline
With each design, the method of controlling the tracking method is specified as color detection. As described in fiducials (section: \ref{sec:detect}) there is specific pros and cons for each method of tracing, and both can basically be implemented in each of the three designs and have the same specified functionalities as the other.

\item Controller functionalities.\newline
All three design ideas are developed with the specific controller functionalities in mind, and does therefore include all of them. The controller functionalities being steering left and right, Acceleration and braking/driving reverse, gear shift and an optional action button for utilities.

\end{itemize}

As described in the previous delimitation, all the design ideas satisfy the list of requirements to the same degree, therefore each design could be used to answer the final problem statement. The implementation will be based on design \# one based on an internal poll.
