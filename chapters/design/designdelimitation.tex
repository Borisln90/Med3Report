\subsection{Design delimitation}

To delimit the three designs, to the actual design that will be implemented, each design will be compared to the list of requirements. Here the design ideas will be evaluated in terms of how well each specific design fulfil the requirements and therefore might answer as a possible solution to answer to the final problem statement. See section \ref{LOR} for the list of requirements

\begin{itemize}
\item The controller must be for controlling vehicle games.\newline
Each design is specifically developed with vehicle games in mind. Therefore no specific features of any of the ideas is relevant in terms of this specific requirement.

\item The controller must utilize visual computing within the limitations of a webcam.\newline
Each design idea is based on a standard webcam. No design idea has any special requirements beyond those defined in section \ref{webcam}. 

\item The controller must be comparable to pre-existing controllers in terms of functionality.\newline
All of the design ideas are developed based on the commonly utilized controls as researched in \ref{gamecontrols}. Therefore this requirement is fulfilled by each of the three design ideas.

\item The solution must require a standard webcam.\newline
All three design ideas is based on utilizing a standard webcam, except design 2 which requires setting up the webcam in a specific way.

\item The solution must require a computer to run the software.\newline
All three design ideas is based on utilizing a computer to run the software.

\item The standard webcam does not have to be embedded in the computer. Therefore computer type is irrelevant.\newline
None of the design ideas specify additional requirements to the computer that should be used, therefore this requirement is fulfilled by all three design ideas.

\item No additional hardware required for the sake of control of the vehicle game.\newline
The design ideas specifies that computer hardware is required. Additional items are accessories which must be utilized to control the software on said hardware, and does therefore not count towards additional hardware. This is the case for all three design ideas. Note however that some might consider webcams as being additional hardware.

\item The controller software must be compatible with a pre-existing vehicle game which supports
the game functionalities as described in Controller functionalities.\newline
Each design idea simulates keyboard buttons as output to each game. This means that the controller should be compatible with vehicle games that might deviate in control scheme.

\item The output of the software should mimic input to conventional input methods like keyboard
presses or mouse movements.\newline
The design ideas simulates keyboard presses to send commands to the game. 

\item The controller software must track either colors, patterns or both.\newline
With each design, the method of controlling the tracking method is specified as color detection. As described in fiducials (section: \ref{sec:detect}) there is specific pros and cons for each method of tracing, and both can basically be implemented in each of the three designs and have the same specified functionalities as the other.

\item Controller functionalities.\newline
All three design ideas are developed with the specific controller functionalities in mind, and does therefore include all of them. The controller functionalities being steering left and right, Acceleration and braking/driving reverse, gear shift and an optional action button for utilities.

\end{itemize}

As described in the previous delimitation, all the design ideas satisfy the list of requirements to the same degree, therefore each design could be used to answer the final problem statement.
As per requirement given the project structure, it is a needed delimitation that a single design concept is chosen to take into further development. Based on the above mentioned section it can be established that all designs are equally capable of containing the requirements and ultimately answer the final problem statement. There are, however, both strengths and weaknesses in the design, and these have been discussed internally. The following section will account for the choice of Design idea 1. See section \ref{design1}.	

Design idea 2 (section \ref{design2}) differs from the other two design concepts due to the requirement of an external webcam and a mean to attach it above the gaming surface. This might ultimately go against the project concept that states that the artifact is required to be easily accessible and usable by most. Therefore it is estimated that more integrated webcams are predominantly accessible by most of the target group. For this reason the design idea 2 has been downsized. Design idea 3, see section \ref{design3} for description, utilizes the integrated webcam; however this design concept does not include gestures, which mean that BLOBs will have to be used instead to cover the requirement of functionality that must be comparable to SOTA-controllers. The BLOBs will have to be situated on the opposite site of the race-wheel, which ultimately can make it harder for the user to see where he/she is placing the fingers to block the BLOBs. Using BLOBs as buttons is risky, since electronic buttons will always have an advantage in this regard. Additionally the specified numbers of functionalities might cause plausible difficulties when several BLOBs are placed upon the controller. Thus making it likely to accidentally activate several functions, or making it generally difficult to manage said controller functionalities. Currently, the only disadvantages of design idea 1 is that it requires more thorough image processing to work, and might be more time consuming, and also more complicated for the user to learn, since an implementation of several functionalities will have to be implemented. The final decision was ultimately between design idea 1 and 3, and given the comparison between the cons, it was established that design idea 1 will potentially render the most positive result, and make room for customization of the design-content if certain approaches are found to be an impasse. 
