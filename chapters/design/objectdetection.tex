\subsection{Alternative object detection method}
For all of the above three designs the method used to acquire input from the user, is by detecting one or multiple colours, either moving around in the scene or merely the presence of it.  One problem that can occur when sticking to this method is, as described earlier in the analysis (section: \ref{sec:detect}), that falsely detection of an object can occur, if the user, or any other unimportant object for the camera, contains the same or a similar colour as the important object. Not only can this be a problem when there are two similar colours in the scene, but also when the lighting conditions in the scene changes.
\bigskip

For a prototype based on the above designs (or just one), it would be feasible to stick to the simple colour detection method. This, however, requires that when evaluating the prototype, factors known to cause trouble with the detection, has to be eliminated to get the most reliable results when evaluating on factors that are not related to detection reliability. On the other hand, if the colour recognition method turns out to be unreliable, it would be beneficial to turn to template matching/pattern recognition using the described fiducial markers \parencite{Fiala2005}. If this change in the design should be applied, it has to be considered to what extend the change should be. This means, that the time it takes to process each frame would most likely be increased and therefore it has to be considered how big an increase in processing time is reasonable.