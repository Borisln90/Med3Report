\subsection{Evaluation discussion} \label{sec:evaldisc}
In accordance with the definition of the problem statement, the goal of the evaluation was to obtain data which would prove or disprove the final problem statement and additionally elaborate on the content to analyse, to what extend the questions has been successfully answered. 
Also, to gain knowledge of specific elements, being the list of requirements, which was successful and the ones that was not.

The results that were acquired were qualitative data from the observations and interviews and quantitative data that were acquired during the observations. 

\subsubsection*{Procedure}
What cannot be interpreted from the results, since they are not accounted for, but can plausibly result in bias are the general methods for the procedure of the test. 
During the tests of the test subjects, other pending subjects were waiting in the same room as the currently testing subject.
By being witness to the test and the general procedure, it is likely that the pending test subjects could form opinions, prepare for the pending test and plausibly contribute to form opinions or preparations. 
Thus it would result in the pending test subjects to be biased, to knowledge of the procedure prior to experiencing it themselves.

\noindent Additionally, the current test subject would be aware of the attention that him/her would be subject to, and therefore cause bias as plausible human relations would cause unforeseen bias. These test conditions can however not be observed in the test results, as these conditions were ongoing throughout the test.


\subsubsection*{Procedure}
The results from the observation notes are qualitative interpretations of the observations that were made during gameplay from the test subjects. 
What was accounted for was the data that was deemed relevant by the two group members that was responsible for observing the ongoing gameplay.
\bigskip

The results that were gathered, as handwritten notes, were processed and elaborated after the completion of the tests, to ensure understanding and content of the acquired data. 
The data was subject to analysis from grounded theory, which made it possible to give clear indications of the amount of data that would indicate that the observed results and notations actually were correct. 
The undergoing process of elaborating simple notations, to an estimate of conclusive results has several elements, which can cause bias such as misinterpretation, incorrect observations; lack of data that results in plausible data having meaning, which is different from the noted definitions and so forth. 
\bigskip

The method and procedure for observation was to note down any significant differences or difficulties during gameplay, or compared to the gameplay of the other controller, that the same test subject completed. 
Underlying the procedure, it was plausible that the two observers noted down the same observations and therefore missed gameplay while writing down the observation. 
Additionally since the observations were written down, the notes was plausibly short, to the bone, and could therefore likely be misinterpreted at a later point, or lack information as to what caused the observation to be noted.
\bigskip

Also, since the gathered data is qualitative, it can arguable cause bias as the obtained data can be misinterpreted all depending on the person whom interpret and understand the data.
However, by gathering notes from several test subjects and undergo the process of grounded theory, it made it possible to acquire patterns within the data to validate the observations, as to what occurred and what might cause the incident. 
\bigskip

The method of acquiring qualitative data, by writing down the observations, made it very hard to elaborate the obtained data.
Therefore to eliminate bias, and to document and account for the observations, it would have been highly relevant to combine the process with footage of the gameplay to validate the observations by comparing the two. 
Additionally, the accounted data was limited to only containing what the two group members thought to be of most importance at that specific moment. 
Therefore it could also be hard to validate later on, since no additional acquiring of data was occurring. 
\bigskip

To summarize the general bias from the method of gathering qualitative data by observation was that the interpretations of the observations were very personal, as the observer was the one who, alone, decided upon what was relevant and why. 
And under the process of describing the gathered data, as it is qualitative, it can plausibly be misinterpreted as well as misunderstood by the observers. 
However the data indicates that by adding the observation notes of the two observers, there is a general consistency of what was deemed relevant, thus implying that the data is relevant and contains valid results.

\subsubsection*{Grounded theory}
The results from the observations are obtained using grounded theory, thus allowing the possibility to rearrange the data acquired and sort and interpret the data into categories to construct a generalized interpretation of the qualitative data.
Since the categories, which has been used to place the data into is defined by the group, personal thoughts and opinions can plausibly cause bias. 
The method is however efficient as it makes it possible to collect data with similar content, thus to some extent, validating the observations, as it become apparent that the noted observations are recurring with several test subjects when playing with any of the controllers.

\subsubsection*{Product and SOTA results and comparisons}
The acquired results which were obtained from the observations were based upon difficulties, with the functionalities of the controllers. 
Generally, there were no observations with the SOTA controller that indicated any sort of technical difficulties. 
During the notations from the observers, the lack of comments to the SOTA controller observations is a result of the opinions from the observers, that no real problem with the functionality with the SOTA controller was present or occurring.

Thus comparing the product observations and SOTA observations, it becomes clear that the functionalities of the product controller have been implemented, but have several function complications which must be improved, to tangent the success criteria of the SOTA controller.

\subsubsection*{Interviews}
The interviews consist of interpretations of the documented answers, and included the test subjects’ opinions to the implemented functionalities and what might be wrong, or plausibly improved. 
The interview is also qualitative, and therefore falls under the plausibility that bias can occur with the understanding and interpretation of the data. 
Also, during the questioning the questions often had to be elaborated in order for the test subjects to understand the terms and purpose of the questions, thus resulting in plausible leading questions, without the intention of doing so.
However, with the interviews, and the interpretations of them it became apparent that similarities of the observed notations and the responds from the interviews is sharing numerous indications as toward what needs to be improved or modified. 