\subsection{Usability test}
After setting up for the test, the group conducted a final usability test, in order to ensure that everything worked as intended, and no elements of the testing room interfered.
This was done by having group members conduct the test a few times, leaving the data collection out. 
The group systematically tested each of the functions as thoroughly as possible, and in doing so, discovered a few issues.

\subsubsection{Problems with gear shift}
During the usability testing, some problems were encountered with the gear shift function.

The gear shift was supposed to be activated when the BLOB, marked during the setup, were moved within a box on the screen - the upper part should shift the gear up, the lower part should shift it down, while the center would be ignored, giving the user room to "enter". 
This, however, proved to be an issue - as the gestures required to execute the function were rather difficult to perform, while also maintaining control of the steering wheel. 
Additionally, unresponsiveness were encountered on several occasions - even with a second person handling the gear shift functions (to eliminate the possibility of the problem being caused by the difficulties of activating the function), it seemed to activate irregularly i.e. the gear would sometimes shift up, when trying to shift down, activate on its own, or not at all.


So with testers signed up for testing within the hour, the group decided to exclude the gear shifting all together, ensuring the test was working at the scheduled time period, rather than postponing the test, risking not being able to get enough testers. 
This course of action removes a required element from the software product though, and as such, it is something which should be reworked during the redesigning phase.

\subsubsection{Camera shift difficulties}
Some problems were encountered, when the group stress-tested the program by trying out the camera shift action, while accelerating and turning, namely that one of the inputs would simply be ignored - It became apparent that a maximum of two inputs could be activated, therefore the camera change function were changed slightly - it now requires the wheel to be in a neutral position, which therefore doesn't allow the speed to be in- or decreased while changing cameras, which were deemed a reasonable compromise.


During the actual testing, it became apparent that the limits, which marks the area where the camera view will not be changed, were perhaps slightly too small, as many people accidently changed camera view, while trying to steer due to the fact that the wheel had no center attachment, and as such would be moved around, passing the boundaries on several occasions.