\subsection{Test content} \label{sec:testcontent}
Here, the strategy of the methodology of the test will be described. 
The specific data acquiring methods, test setup, description of test approach, data sheets and the structure of the general test procedure.

\subsubsection{Goal}
The goal with the test is to accumulate data which can ultimately prove or disprove the problem statement and the list of requirements. As specified in the method chapter (\ref{sec:hypotheory}) a hypothetic test will be conducted with an experimental structure for which it is possible to define variables which can be investigated. Additionally, independent variables can be researched to account for different data outputs.


The variables cover the following data, for which outputs will be obtained:
\begin{itemize}
	\item Final problem statement
	\item List of Requirements
	\item Controller Functionalities
	\begin{itemize}
		\item Steering left and right
		\item Acceleration
		\item Braking and/or driving reverse
		\item Gear shift
		\item (optional) action button
	\end{itemize}
\end{itemize}

Additionally, the test will allow the users to provide feedback thus enabling the possibility to get data that reflects the level of success for which the product has been developed and implemented. 

\subsubsection{Target group}
Non probability sampling will be utilized. 
It is a subset of people whom represent the general population of people that is familiar with vehicle games as defined in the method chapter (\ref{sec:approach}).
A preferred amount of 20 test subjects will allow for a diverse sampling for which, plausible data can be triangulated to allow for validation in terms of the acquired data.

\subsubsection{Test materials}
The test will be conducted with the use of the game Dirt 3. 
Dirt 3 is a rally simulation vehicle game that supports both the choice of sota controller as well as keyboard input that is needed for the system to simulate.  
The game has all the functionalities as specified in the list of requirements, and can therefore serve as a platform to test the controller.
To compare the product with the SOTA controllers, an Xbox 360 controller will be utilized to allow comparisons of the two.


\subsubsection{Delimitation of SOTA Controller for the test}
In order to ensure that the data gathered from the testing of the group's product actually refers to the product, and not to the game or other non-important elements of the test, a control experiment will be conducted.

The purpose of this control experiment is to isolate relevant data. 
This is done by conducting an exact replica of the experiment, with the exception of the controller used - in other words, the only variable of the two experiments, is the controller used. 
In addition to isolating data, the control experiment allows for a comparison between the custom product being tested, and the SOTA controller used in the control experiment.
\bigskip

To determine which SOTA controller to use for the control experiment, a few points should be taken into consideration:
\begin{itemize}
\item The SOTA controller must be naturally compatible with the game used for the test, in order to eliminate the risk of bias being created from mismatch between game and controller.

\item The SOTA controller must be within the available means of the group.
\end{itemize}

While an Xbox Kinect might be a more direct comparison, in terms of how the game is played, it is not necessarily the best option as, as per the above pointers, the controller should be naturally compatible with the chosen game, which the Kinect is not. 
The Xbox 360 controller, however, fulfils the requirements for a comparable SOTA controller, and since there was one available to the group, it was chosen for the test.

\subsubsection{Observation}
While playing, the test subject(s) will be observed. 
The following observations will be made throughout the test subject’s gameplay in order to account for similarities or differences of the two controllers. 
Functions are defined as the controller functionalities. 
The observations are based on noticeable events and does not cover log of the data thus it will be accounted for if the test subject is experiencing any of the following function complications.

\begin{itemize}
	\item Function delay\newline
		The general response rate from the subject enabling a specific function, to when the actual controller function is 				activated.
	\item Non responsiveness of functions\newline
		The response for which, utilizing a function is either non responsive or behaving anyway that differs from the SOTA 			controller which complicates the gameplay.
	\item Activation of functions\newline
		The activation of any controller functionality which is not activated when desired at any point.
	\item Noticeable complications when enabling consecutive functions.\newline
		Complications that might occur while enabling several controller functionalities in a consecutive order or 						simultaneously.
	\item Noticeable difficulties when handling and utilizing the controller.\newline
		If the test subject experienced any problems with the usage and handling of the controller in terms of movement and 			gestures.
\end{itemize}


\subsubsection{Interview} \label{sec:interview}
An interview will be conducted after the consecutive game tests. 
An interview is utilized because of the use of open ended questions which will allow the users to give detailed answers, as opposed to questionnaires with open ended questions which might result in very limited response.  
The interview will cover the following subjects.

\begin{itemize}
	\item Questions that will substantiate the observations.
	\item Questions that will account for unforeseen opinions, comments or thoughts from the test subjects.
\end{itemize}

 For the full list of questions see (XX APPENDIX QUESTIONAIRE).
 
 
The interview will plausibly indicate whether or not the observed notations are applicable for the accounted observations.
This means that the specific observed complications will be compared to the test subjects own opinion of the experienced plausible complications thus proving or disprove the accuracy of the observation. 
Additionally, the interview will allow the test subjects to give detailed information towards plausible thoughts and opinions of the product in order to evaluate the success of the controller in terms of requirements and ultimately answer to the final problem statement.


\subsubsection{Test setup}
The method for the test is conducted with the structure of a within-group design as specified in the method chapter \ref{sec:hypotheory}.
\bigskip

The test will be divided into two separate test phases for which each subject will be asked to conduct each one consecutively. 
Thereafter an interview will be conducted. 
Each test subject will either start to play with the developed product or with the SOTA product. 
After each test, the sequence will switch, to gain varied data of the two controllers to minimize plausible bias, by starting with one specific controller for each test.
\bigskip

The second test phase will be the user playing the same track(s) in Dirt 3 where the other controller is utilized. 
Here the user is able to draw comparisons between the two controllers thus making it likely to obtain data of the plausible similarities or differences of the two controllers.
\bigskip

Lastly as specified in interview(\ref{sec:interview}) an interview of the test subject will be conducted. 
