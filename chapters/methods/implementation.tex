\subsection{Implementation}
In order to create a prototype that can be used for testing with satisfactory results, as well as serve as a baseline for future development, the final design will be implemented as a horizontal prototype, where as many plausible functionalities can be tested as possible. Horizontal prototyping as opposed to vertical prototyping is based on the idea of function before aesthetics \parencite{Rogers2002}. Horizontal prototyping means creating a prototype that has a wide range of functionality but in a very crude state. Vertical prototyping provides a lot of detail but only on a few set of features. This means that in relation to this design a software prototype will be created that will contain, close to, the full range of functionality from the final design but in a crude state that will not be user friendly or meet the performance requirements. 

The implementation will consist of a single piece of software that will map the input provided by a webcam to a set of keyboard and mouse commands. This software will work in the background and provide input to any application intended for testing.
