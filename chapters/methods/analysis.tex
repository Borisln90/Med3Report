\subsection{Analysis}
At this point, it has been made apparent which subjects that should be focused on from the Naïve designs. The analysis is a simple, yet time-consuming process to execute. The group got to research each focus point and get the information required in order to create a final problem statement (FPS) that will fulfil our requirements to this project. The focus points in this analysis will be:
\begin{itemize}
\item What have others done before us? \newline
State of the art(SOTA) will be researched to experience what others 	have created, to learn from their mistakes and to find inspiration for our projects and to find alternatives to already proposed solutions. 

\item How does image processing work? \newline
Image processing will be researched in order to be able to understand how the process works and the information that has been learned, to create a product that is functional and capable to produce data that can be used for evaluation.

\item Can we form our gestures so the volunteers feel comfortable during tests? \newline
If results is to be utilized at a later point for which a test can be conducted, in terms of, feeling comfortable and utilizing natural gestures.

\end{itemize} 
\pagebreak[1]
When these steps have been followed, a final problem statement can be defined an formulated. Thereafter, the final problem statement will be used to create new designs based on the new information gained in the analysis and then be able to move forward to the design phase.
