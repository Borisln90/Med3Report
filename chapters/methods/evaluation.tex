\subsection{Evaluation}
It is essential that the methodology of the evaluation for the product-test is defined in terms of how the test must be conducted along with how the answers will be interpreted and concluded upon. To achieve this, the desired outcome of the test must be clearly identified i.e. a proper strategy for evaluating whether or not, and to what degree, the final problem statement has been answered. Furthermore, the relevance of the established list of requirements must be tested.

The main criterion of the product test can be defined thus in bullet points:
\begin{itemize}
\item To investigate whether or not the final problem statement has been answered.
\item To prove/disprove the relevance of the list of requirements.
\end{itemize}

\subsubsection{Triangulation}
(TO BE CONCLUDED)
\begin{quote}
\textit{Triangulation involves using combinations of techniques in concert to get different perspectives or to examine data in different ways.} \parencite{Rogers2002}
\end{quote}

\subsubsection{Usability Test}
\begin{quote}
\textit{“Ideally, user-based testing would take place during all stages of development, but that is not always possible. Why do we do usability testing? As much as designers try to build interfaces that match the needs of the users, the designers are not users and even the users themselves sometimes cannot clearly identify their interface needs.”}
\parencite{Lazar2010}
\end{quote}
\begin{quote}
\textit{"Many people say that five users is the magic number and that five users will find approximately 80\% of usability problems in an interface (Virzi, 1992). This has become a generally accepted number in human computer interaction, but many other researchers disagree with the assertion. The major challenge in determining the right number of users is that you don’t know in advance how many interface flaws exist, so any estimate of how many users are needed to find a certain percentage of interface flaws is based on the assumption that you know how many flaws exist, which you probably don’t."}
\parencite{Lazar2010}
\end{quote}

\subsubsection{Preliminary Test}
(TO BE CONCLUDED)
