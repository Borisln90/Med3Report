
\subsection{Naïve designs}
The naïve designs are, as the name implies, designs based on assumptions and ideas, instead of actual research or considerations concerning feasibility taken into account. The designs are not meant to be used as a final design, but rather as inspiration source for which areas needs further research, before finalizing the design.

The research which will be considered relevant is the areas of research which proves to be indispensable or in any way plausible to be of relevance to the project. Ultimately, finding the areas of research based on the content of the naïve designs, will help to clarify or in any way specify how to create a product.
\bigskip

The designs start as a brain storm, revolving around ways in which the initial problem statement \textit{could} be answered. The ideas from this brain storm, will then be used to create some design concepts, explaining how a product can be created, how it is intended to work and how it is used to answer the problem. Once the designs are made, thought will go into which elements needs further research, in order to actually make the product described, and to make sure that it \textit{actually} tests, what it is supposed to.
\bigskip

This approach of "Initial Problem Statement -> Naïve Design -> Analysis -> Final Problem Statement" replaces the "traditional "Initial Problem Statement -> Investigation -> Final Problem Statement -> Analysis", and the reason this is done, is an attempt at avoiding the risk of getting tunnel vision of a specific way to solve a problem - Instead of focusing on a single way of solving a problem right from the get go, this approach revolves around having multiple solutions in mind, right up until the point, where the \textit{best} (potentially, at least) solution have been found, through the process of delimitation. 

By having several solutions to the specific problem, the goal is to either choose or combine aspects of the design ideas to conduct a general tender to the solution. By allowing several research areas to be defined, by usage. NOT YET DONE
