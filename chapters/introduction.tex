\section{Glossary}
\begin{table}[!htbp]
\begin{tabular}{p{2in} p{3.4in}}
Physical artefact & Refers to the physical components of the custom built controller - i.e. the cardboard "steering wheel" and the gear shift sponge.\\
 & \\
Software program / product / prototype / controller & Refers to the "product" of the project - i.e. the software for detecting movement, and sending data to the game.\\
 & \\
SOTA controller & In the case of the test and evaluation, refers to the XboX360 Controller, used as the SOTA comparison.\\
 & \\
System (impl.) & Refers to the software system designed for this product only, and not the computer-system.\\
 & \\
The user & Refers to the user of the product - i.e. the test subjects.\\
 & \\
Positions (relating to BLOB and the program) & Refers to the recorded positions of BLOBS within the recorded area of the webcam.\\
\end{tabular}
\end{table}

\clearpage

\section{Introduction}
The video game industry has evolved a lot over the past few decades. Video games are constantly increasing in size, gaining more functionality, and the graphical fidelity reaches new heights with every new release. This evolution is possible because the hardware on which the games are being run is also evolving and getting more and more powerful.
\bigskip

One aspect of the video gaming industry which has not changed considerably since the beginning is the controllers used by the different platforms. All the leading video game consoles use a variation of the same basic design: a hand-held controller with one or more analog sticks and a set of buttons for the user to interact with. Sonys PlayStation has used the same design of the controller since the first PlayStation was introduced. The same goes for Microsofts Xbox. The personal computer (PC) has used the basic keyboard and mouse for controls for many years now and it remains the preferred control method.
\bigskip

This has changed over the past years however, as the living room consoles have developed new ways of interacting with video games. These new methods include the usage of visual computing and one or more cameras to read the user, which means that the user now uses body movements instead of buttons to control the games. This introduces new ways of video game interaction and paves the way for a new breed of video games.
\bigskip

One platform where there is not much focus on controls than the conventional methods, using mouse and keyboard, are computers. The digital input from a keyboard does not translate well into vehicle games like racing games where the vehicles do not go all the way left or all the way right. This means that one might have the opinion of it not being the best experience utilizing vehicle games. Analog sticks on controllers can be better used to fine tune the movement of the car. This has led to the development of add-on steering wheels and analog joysticks which can be bought together with vehicle games and plugged into the PC. However these controls often add cost to the PC game and are generally not easy to manage on the desk in front of the computer once you are done playing the game.
\bigskip

Many PCs these days are equipped with webcams in the screen for use in video conversations so the idea of using these webcams together with visual computing to get input from the user could be transferable from the living room consoles to the PC as an alternate method of controlling these vehicle based games for an experience comparable to those provided by the living room consoles.

\noindent This leads to the following initial problem statement.

\subsection{Initial problem statement}
How can a vehicle-game controller be made from a standard webcam using visual computing, while being comparable with other gaming platforms utilizing movement based functionalities?