\section{Readers guide}
\textit{Dear reader, please note that the structure of this report differs from the average Aalborg University(AAU)-group reports, and thus requires some clarification before the perusal is initiated. 
The structure has been introduced by our supervisor, as an experimental attempt to avoid the impasse that occasionally occurs in the start-up phase.
The fundamental difference between the structures is that the investigation is replaced with a concept known as Naïve Designs. Instead of building the final problem statement upon delimitations (the usual purpose of the investigation is to narrow the focus of the problem, and thus end up with a focused direction), the Naïve Designs is an alternative way of designing solutions to the defined problem statement.}
\bigskip

\textit{\textbf{The definition of a Naïve Design} is a design/concept idea without any core-technical substantiation. 
The concepts are merely developed from a brainstorming session where feasibility is subordinate, since it might cause limitations to the creativity if that is to be considered.	
The Naïve Designs will subsequently be considered in more detailed perspective, and we will critically approach and analyse the design-ideas, including their target groups, strengths, weaknesses and research focus. 
Based on the findings, one idea will be favored and the continuing analysis will relate to the prioritized design.}
\bigskip

\textit{\textbf{The Design phase} will require three different design solutions to the final problem statement, and again only one design is to be chosen for the final implementation. 
Theoretically this will encourage the creativity when two additional designs has to be drafted, and any viable ideas from the three different designs can and should be implemented into the final design.}
\bigskip

\textit{A project-report usually ends with an evaluation that reflects upon the test results, and concludes on the success, faults and failures of the design and implementation. 
Seldom these evaluation ends in recreation and reconsideration of the concept. 
In this report we will, after the evaluation, reconsider our final design idea, and based on the data gained from the test results provided, generate a new theoretical re-design.}
\bigskip

\textit{Thank you for showing interest in our research, and we hope you will enjoy our report as much as we enjoyed making it.}
\bigskip

\noindent\textit{Regards,}

\noindent\textit{Group 307}

\clearpage

\section{Glossary}
\begin{table}[!htbp]
\begin{tabular}{p{2in} p{3.4in}}
Functionality & Defined as "the range of things that a computer or other electronic system can do" as per \parencite{Macmillan2005}.\\
&\\
Vehicle game & A genre of interactive video games revolving around the operation of vehicles.\\
 & \\
Physical artefact & Refers to the physical components of the custom built controller - i.e. the cardboard "steering wheel" and the gear shift sponge.\\
 & \\
Software program / product / prototype / controller & Refers to the implementation of the final design.\\
 & \\
SOTA controller & In the case of the test and evaluation, refers to the Xbox 360 Controller, used as the SOTA comparison.\\
 & \\
System & Refers to the software system designed for this product only, and not the computer-system on which it runs.\\
 & \\
The user & Refers to the intended user of the product.\\
 & \\
Positions (relating to BLOB and the program) & Refers to the recorded positions of BLOBS within view of the webcam.\\
 & \\
 
\end{tabular}
\end{table}

\clearpage

\section{Introduction}
The video game industry has evolved a lot over the past few decades. Video games are constantly increasing in size, gaining more functionality, and the graphical fidelity reaches new heights with every new release. This evolution is possible because the hardware on which the games are being run is also evolving and getting more and more powerful.
\bigskip

One aspect of the video gaming industry which has not changed considerably since the beginning is the controllers used by the different platforms. All the leading video game consoles use a variation of the same basic design: a hand-held controller with one or more analog sticks and a set of buttons for the user to interact with. Sonys PlayStation has used roughly the same design of the controller since the first PlayStation was introduced. The same goes for Microsofts Xbox. The personal computer (PC) has used the basic keyboard and mouse for controls for many years now and it remains the de-facto control method.
\bigskip

This has changed over the past years however, as the living room consoles have developed new ways of interacting with video games. These new methods include the usage of visual computing, and one or more cameras to read the user, which means that the user now uses body movements instead of buttons to control the games. This introduces new ways of video game interaction and paves the way for a new breed of video games.
\bigskip

One platform where there is not much focus on controls other than the conventional methods, is the PC. The digital input from a keyboard does not translate well into vehicle games, like racing games, where the vehicles do not go all the way left, or all the way right. This means that one might have the opinion of it not being the best experience for vehicle games. Analog sticks on game console controllers can be better used to fine tune the movement of the car. This has led to the development of add-on steering wheels and analog joysticks which can be bought together with vehicle games and plugged into the PC. However these controls often add cost to the PC game and are generally not easy to manage on the desk in front of the PC once you are done playing the game.
\bigskip

Many PCs these days are equipped with webcams in the screen for use in video conversations so the idea of using these webcams together with visual computing to get input from the user could be transferable from the living room consoles to the PC as an alternate method of controlling these vehicle based games for an experience comparable to those provided by the living room consoles.

\noindent This leads to the following Initial Problem Statement (IPS).

\subsection{Initial problem statement}
How can a vehicle-game controller be made from a standard webcam using visual computing, while being comparable with other gaming platforms utilizing movement based functionalities?