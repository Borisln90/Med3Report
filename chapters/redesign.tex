\section{Redesign} \label{sec:redesign}
This chapter is an extension of the evaluation chapter, as it evaluates upon the design, and how it could have been improved, or changed, in order to get a better result. These are the changes that might not have been apparent during the development of the product, or during early testing, but became clear during either the testing, or during the data analysis. The chapter is structured according to the result analysis from both the observations and the interviews.

\subsection{Regarding delay encountered when activating functions}
The data analysis shows that some delay was encountered during the tests, it is not clear, however, if the experience of delay were only caused by actual delay, or if some of it were rather due to perceived delay, caused by the way data is handled by the program, compared to how the controller is used. 
Before redesigning the program, some research should be conducted, to determine the cause of the delay - if it is caused by the program, or is simply perceived delay caused by the controller.
This means that the following approaches should be taken, when redesigning the product, in order to prevent delay when activating functions:
The program needs optimizing, ensuring as little delay as possible - One specific problem encountered related to the queue of commands which, when exceeding two at once, would cause an error, hence this specific problem is a good place to start.
Finally, the difference between the perceived way of using the controller and the actual input should be corrected - Either the way turning is activated should be changed, making it more obvious that any degree of turn will result in "full turn" in the game in other words; the controller should be binary, like the input. Alternatively the inputs should be changed from a binary keyboard-input to a gradient input, able to emulate the actual movement of the steering wheel. In other words; Consistency.

\subsection{Regarding non-responsiveness of functions}
The main issue that can be derived from the data analysis is some people having difficulties with activating certain functions, or having them activating at random. This means that the gestures required to activate the different functions should be reviewed.
The gear shift function, in particular, needs rethinking as it simply didn't work as intended. Likewise the camera shift function should be reconsidered, as it was quite often activated accidently. Finally, as mentioned previously, the steering, the deceleration and the acceleration functions should be reviewed, making them more consistent with how the input is given to the game.

\subsection{Regarding activation of game functions}
The test subjects reported difficulties regarding activation of the game functionalities, which mostly stemmed from the required gestures - Not so much performing the gestures themselves, but rather the amount of movement required to activate them e.g. many people had troubles putting the car into reverse, as they had to pull the steering wheel further back, than what was comfortable.

\subsection{Regarding activating multiple game functions at once}
Most of the problems related to activating multiple functions at once, have already been covered in the previous sections. That being said, however, the fact that the program cannot handle more than two functions at any given time means that the user is limited to two actions at once, which can cause delay or trouble activating multiple functions. Therefore, if possible, the program should be optimized to allow more commands to be executed at the same time.

\subsection{Regarding complications associated with controller movement and handling}
In addition to the already mentioned problems, such as the amount of movement required to reverse the car, many testers also noted that the weight of the 'steering wheel' was slightly too light, making it hard to keep steady, thereby sometimes making them activate functions unintentionally - e.g. the camera shift.
As such, in addition to reviewing the gestures required for the different controls, some more thought should be put into how the physical controller is constructed, ensuring a more realistic and appropriate weight and size.
