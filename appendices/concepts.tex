\section{Concept development and concept into categories} \label{app:concept}
From the categories which have been defined, broader groups of similar concept are identified to form categories of data which is approximately similar in terms of context. 
These categories will be deemed to be of similar content and will include a detailed interpretation of the specific categories.
\bigskip

\noindent\colorbox{NotLavender}{\textbf{Delay activating functions}}\label{eval:delay}\newline
These groups of responds are deemed to be of similar content and describe observations that indicate that either the observer is observing delay or the test subject is expressing it, from where a certain action is activated till the actual desired action is performed.

\begin{itemize}
\item Slight function delay.
\item Slight function delay.
\item Slight delay spikes noted.
\item Slight function delay.
\item Function delay – turning.
\end{itemize} 
\bigskip

\noindent\colorbox{NotGreenYellow}{\textbf{Acceleration, Reverse/braking relevance}}\newline
The following data elements describe issues and difficulties which are similar to the general control of the controller. The relationship between these notations is inability to activate a certain action, loss of control of a certain action or in any way irregular incidents which occur with the above mentioned category.

\noindent\textbf{Controller placement issues}
\begin{itemize}
\item Breaking was non responsive. Wheel was not forward enough to enable it.
\item Problems with correctly placing the controller. Referring to the placement in space, (point of forward/backward etc.).
\item The wheel was placed very far forward, in order to accelerate.
\item No noticeable braking during race
\item Lost control of the forward/backward mechanics - Trouble activating them after the other.
\item Function activations were unstable. Speed/ braking.
\item Sudden stop in acceleration while driving.
\item Random clusters of not accelerating, happens randomly.
\item Tried to brake. Nothing happened. (Subject did not move backward enough).
\item Moves forward/backward, a little, indicating that the subject expects to accelerate more or less depending on how close the subject is.
\item After in game crash, the subject was struggling with enabling reverse and turn to get back on track. Non responsive.

\end{itemize}
\noindent\textbf{Game setting relevance}
\begin{itemize}
\item Controller is moved forward – braking. Noted: Because of the auto steering.
\end{itemize}
\bigskip

\noindent\colorbox{NotSkyBlue}{\textbf{Change camera view relevance}}\newline
The change camera action accounts for the following sub categories and the content of the data is deemed to be of similar content as the mentioned sub categories.
Unintentional activation of the action({\color{Red}3.1}) account for the cases where observations which describes unwilling activation of the said control.({\color{Red}3.2}) specifies the occasions for which successful criteria has been deemed as activating the control has been listed as a category. Lastly, ({\color{Red}3.3}) specifies the data for which the test subject did not try to use the action, or was unsuccessful in trying to.

\noindent\textbf{3.1 Unintentional/accidental/unaccounted activation of the change camera setting}
\begin{itemize}
\item Random activation of the camera view change. Occurred +5 times.
\item Change view happens for no reason.
\item Random change of view occurred 3-4 times.
\item Camera view changed randomly, 3-4 times.
\item Random activation of change view. 6-7 incidents.
\item When subject tries to reverse – view is changed
\item Change camera view is randomly activated. 4 times noted.
\item Change camera function is randomly enabled.
\item Random activation of camera change. Occurred once.
\item Random activation of camera change. Occurred +12 times.
\item Random activation of camera view.
\item Random activation of change view. 2-3 times.
\item Random activation of change view.
\item Accidental activation of view change. (Subject flips the wheel, pointing downward). Occurs 7+ times.
\item Moves the wheel a lot to left/right when turning, therefore change view is enabled.
\item Randomly enabled change of view twice.
\item Controller moved right – None responding activation of change camera view.
\end{itemize}

\noindent\textbf{3.2 Active and intentional activation of the change camera setting}
\begin{itemize}
\item After practice, enabling the “change view” was improved and functional
\item Camera change view is enabled by flipping the wheel over the shoulder.
\item Change camera is enabled several times on purpose during game
\item Change of view was activated once on purpose
\item Camera view was changed twice.
\item Activated the change view function twice.
\item Successfully enabled camera view twice
\item No random activation of camera is occurring.
\item Activation of camera intentional happened twice.
\end{itemize}

\noindent\textbf{3.3 No activation of the change camera setting. Either optional or unsuccessful.}
\begin{itemize}
\item No activation of camera change function
\item Subject did not use camera view option.
\item Tried to enable “change view”. Nothing happened
\end{itemize}
\bigskip

\noindent\colorbox{NotOrange}{\textbf{Turning relevance}}\newline
Turning relevance covers the general steering left/right in game. Additionally, in-game accidents, test subject movements that affected the turning and general comments/actions which relate to any elements that explains turning. ({\color{Red}4.1})In-game steering difficulties: Overturning, covers the data for which difficulties with steering has been observed and noted. ({\color{Red}4.2})  Controller mechanics: Occasional unwilling turning specifies the observer notes for which unwilling turning was happening.

\noindent\textbf{4.1 In-game steering difficulties: Overturning}
\begin{itemize}
\item Several occasions: Car was wobbling. Hard left, or hard right.
\item Subject was driving off the track repeatedly.
\item Uncontrollable steering during most of the lap.
\item Could only turn a lot. Small turns were over exaggerated and therefore drove off the track.
\item When turning, only big turns were possible. Always too much left or right.
\item Overturning a lot therefore ends up off the track a lot.
\item Turning is only exaggerated, making handling of the vehicle difficult.
\item Over turning, small turns were hard.
\item Driving off track because of large turns.
\item Lots of big turns, ending up off track.
\item Only big turns is happening. No small turns difficult.
\item Only “big” turns are possible. Small turns are left to the auto steering.
\item Over steering noted.
\item Small notions (movement) did not do anything.
\item Ignoring small turns, since turning would result in over steering. (Commented while playing).
\item Turning was exaggerated.
\item Over exaggerated the turning motion to turn.
\item Subject only tried turning during “big” turns
\item Left small turns to the auto steering.
\item Controller only reacts to big turns.
\item Over steering a lot.
\item Over turning. Losing control over the vehicle
\item Wobbles a lot.
\item Over turning noted.
\end{itemize}


\noindent\textbf{4.2 Controller mechanics: Occasional unwilling turning }
\begin{itemize}
\item Incidents of unwillingly turning right.
\item Vehicle is occasionally turning right.
\end{itemize}


\noindent\colorbox{NotRed}{\textbf{Subject conditions, comments \& actions}}\newline
The subject conditions, comments \& actions covers the data for which observations towards postures, positioning, actions of the test subjects or conditions have noticeable relations. 

\noindent\textbf{5.1 Observed postures of the test subjects during gameplay.}
\begin{itemize}
\item Arms of subject are placed very low. Near thighs in the sitting position.
\item Kept arms straight, (wheel close to screen) while driving.
\item Arms are held very low.
\item Kept the controller in straight arms – aiming it close towards the computer.
\item Arms are held low, close to thighs
\item Subject’s arms are held low, close to thighs sitting down and in straight arms – close to the camera.
\item Subject holds the wheel in straight arms towards/close to the camera.
\item Wheel is held in straight arms.
\item Subject is keeping arms straight, close to the screen.

\end{itemize}


\noindent\textbf{5.2 Subject comments during gameplay}
\begin{itemize}
\item Subject quote: “shoulder hurts”
\item Subject note: Very “binary” control.
\item Subject note: requires adaption to control.
\item Subject quote: “I would have rage quitted long ago”.

\end{itemize}


\noindent\textbf{5.3 In-game steering recovery observations}
\begin{itemize}
\item Wheel (board) was tilted, excessively when car was out of control (things were going bad).
\item Noticeable frustration of the test subject.
\item After in game crash, subject had difficulties getting control over the vehicle, getting back on track. Forward/backward \& turning.
\item Wheel board is tilted when subject is experiencing some unexpected behavior of the vehicle. ex. Over turning.

\end{itemize}


\noindent\textbf{5.4 Observed plausible clothing difficulties}
\begin{itemize}
\item Subject is wearing a red shirt and might interfere with the color registration.
\item Note: subject wearing red shirt. Might cause complications.

\end{itemize}



\noindent\colorbox{NotGreen}{\textbf{6.	Game \& in-game crashes}}\newline
Game \& in-game crashes covers the crashes in-game, which is defined as driving off the road, losing control over the vehicle. Game crashes indicates difficulties with the game itself, being the developed software or the actual game which suffered temporarily conditions, deemed defined as a crash.

\begin{itemize}
\item No in-game crashes.
\item Game was reset because of technical difficulties – not associated with product.
\item In game crash noted.
\item Game was restarted. The gear could not get above 1st. Settings was sat to manual.
\item In game crash noted
\item In game crash noted.
\item Noted in game crash.

\end{itemize}



\noindent\colorbox{NotPurple}{\textbf{Frame rate relevance}}\newline
Frame rate relevance covers the incidents for which a significant drop in frame rate during the laps was observed.

\begin{itemize}
\item Drop in frame rate during lap for several seconds.
\item Drop in frame rate for about 20 seconds.
\item Drop in frame rate is observed.
\item Decrease in frame rate. About 10 sec.
\item Decrease in frame rate is noted.
\item Drop in frame rate.
\item Drop in frame rate.
\item Frame rate reduction. About 5 sec.
\item Drop in frame rate for seconds.

\end{itemize}