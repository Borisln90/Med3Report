\appendix
\appendixpage
\addappheadtotoc

\section{Naïve Designs} \label{NaiveDesigns}

\subsection{Control a car through gestures} \label{nd1}
\subsubsection*{Naïve Design}

\begin{figure}[h]
\centering
\includegraphics[width=5.5in]{NDesign1}
\caption{Concept illustration of idea 1}
\label{fig:ndesign1}
\end{figure}

For this design a standard webcam found in most laptops and other types of cameras that can send images to a computer, will be used. The idea is to make a vehicle controller, out of a camera, that can react to a user’s hand gestures and translate that into input for a vehicle based application or game. These gestures would translate to tasks commonly used in relation to vehicles, like speed up, slow down, change gears, hand breaking.


The idea here is to use hardware to get input from the user, however this hardware should not be specialized for gesture recognition, like a Kinect or Eyetoy. The hardware should be common to most computer users. The user should only need to install a piece of software and be ready to use gestures.


To make the camera recognize the different gestures some software will have to be implemented. This software will need to recognize the outline of the user and translate the gestures made by the user into commands for an application. The software should be able to recognize both static and moving gestures, for example, a static, flat hand shown to the camera could translate to a handbrake, or a fast movement of the right arm could translate to a hard turn to the right.

\subsubsection*{Design Analysis}
\noindent\textbf{Motivation} \newline
The motivation behind this concept is to replace the standard ways of using controllers with methods that does not require expensive specialized hardware to operate.

This can be used to control applications like driving games without the need for specialized driving wheels, or analog joysticks, requiring only a basic set of hardware already available to most people who own a laptop. 
\bigskip

\noindent\textbf{Strengths and weaknesses} \newline
Strengths:
\begin{itemize}
\item This can make a driving game more approachable.
\item This can make a driving game controller less costly.
\end{itemize}
Weaknesses:
\begin{itemize}
\item Precision is very important. We do not want to end up in a ditch.
\item Input is not ergonomic.
\item High input latency becomes a much bigger concern.
\item Input methods are not intuitive.
\end{itemize}
\bigskip

\noindent\textbf{Target group} \newline
The primary target group for this concept is people who are confined to using specialized hardware for certain applications, like driving games or equally like-minded applications.
\bigskip

\noindent\textbf{What do you know / need to know} \newline
For this concept a good understanding of image processing is key. The product will be based on image processing to get its readings and translate into input. Precision is key to this product because it will potentially be used for precision operations like navigating a vehicle around a sharp turn.

Knowledge is needed on how to translate data from a camera into input that the supported application(s) can understand and use.
\bigskip

\noindent\textbf{What to investigate} \newline
The main focus of the research is to figure out how much of this is possible and how much of it is possible using a standard webcam. Investigation is needed on whether it is possible to get accurate readings from a camera that is reliable enough to be usable and also whether the ergonomics of this approach is sustainable.
\bigskip

\noindent\textbf{Initial prototype / mock – up.} \newline
An initial prototype could involve using markers on the hands of the user and use one or more cameras to capture data. Software will be needed to translate the input, as well as a demo application involving navigation will be needed. This prototype will be used to answer what is possible with this concept and what needs improving. 

\pagebreak

\subsection{Controller blocks} \label{nd2}
\subsubsection*{Naïve Design}

\begin{figure}[h]
\centering
\includegraphics[width=5.5in]{NDesign2}
\caption{Concept illustration of idea 2}
\label{fig:ndesign2}
\end{figure}
This idea would give the possibilities to both play as a single player and multiplayer (coop and versus). Each player would have two blocks to control their car in the game. By moving the blocks around, the camera can detect their placement and then use this information to control the car in the game. If the blocks are placed and maneuvered on a table, a camera would be looking down at the table from above. In the top right of figure \ref{fig:ndesign2} is illustrated how the camera could split up the playing field (where the users are moving the blocks around in) and in this way distinguish between which blocks control what car.

The control blocks would work e.g. as follows:
\begin{itemize}
\item Both blocks forward: the car accelerates
\item Both blocks back: the car brakes
\item Left block forward + right backwards: car turns to the right (and vice versa)
\end{itemize}
For other games that are not car racing, the blocks could be coloured. With this addition, the blocks could be rotated, which the camera could register by the colours on the blocks.

\subsubsection*{Design Analysis}
\noindent\textbf{Motivation} \newline
\noindent\textbf{Problem} - This idea is based on a way to make a gaming platform cheap, compared to other professional and popular platforms such as the Nintendo Wii, and Xbox Kinect, which both tracks movement performed by a user.
\bigskip

\noindent\textbf{Opportunities} - The idea here would give the possibility to play together (or against) your friends, as you can play multiple players at the same time. You can cooperate as two players in controlling one vehicle in the game, if you would like. This possibility is obtained by using two or more blocks on the board to control each vehicle.
\bigskip

\noindent\textbf{Inspiration} - The source of inspiration was the “music board” where you are using different kinds of blocks, and by placing them on a surface, you create different kinds of sounds. 
\bigskip

\noindent\textbf{Strengths and weaknesses} \newline
Strengths:
\begin{itemize}
\item Options for multiplayer gaming as well as single player
\item Cheap: only a webcam is needed, the blocks can be homemade
\item The blocks can both be moved around and rotated, giving several ways of controlling the vehicle
\item Social gathering, as the idea is to gather more people around the table and play together in the same room.
\end{itemize}
Weaknesses:
\begin{itemize}
\item Camera placement can be a bit problematic: the camera has to see the blocks from above
\end{itemize}
\bigskip

\noindent\textbf{Target groups} \newline
As the main problem is focused on making a gaming controller of some kind, the target group will primarily be people already playing video games. This does not necessarily mean hard-core gamers. In fact, non-hard-core gamers would be of preference. Initially this idea might seem more appealing to the same kind of people that uses Nintendo Wii, Xbox Kinect or alike, as this idea involves having people in the same room when playing. Those kind of people might be families playing together once in a while, or friends gathering to play together in another way than LAN parties.
\bigskip

\noindent\textbf{What do you know / need to know more about?} \newline
\noindent\textbf{Visual computing} - Some knowledge needed to make this idea come true would be how visual computing actually works. The camera in this idea would need to recognize the blocks on the table and read how they are being manipulated to perform an action in the game. 
\bigskip

\noindent\textbf{Users} - The product of this idea especially has to be appealing to the target group in order to give them the satisfaction they would have gotten from using a popular gaming platform. This means that the controls have to be easily understandable and manageable as well as many other things, which would need research.
\bigskip

\noindent\textbf{SOTA} - At the moment it is unknown if there are ideas like this already developed. So far it is known that there is a device that utilizes something like this idea, but instead of controlling a vehicle in a game, the device provides sound and music according to what kind of block you place on a surface that reads the blocks.
\bigskip

\noindent\textbf{What would you be investigating?} \newline
\noindent\textbf{Image analysis and processing} - 
Using a webcam will provide a flow of images of the area in which the users will be controlling their blocks. These images need to be analyzed and processed in order for the software, that will be developed, to understand what the user wants the program to do. To be able to use the images like this research has to be done in the area of image analysis and processing.
\bigskip

\noindent\textbf{Initial prototype} \newline
For an initial prototype, it would be useful to focus on using one specific video game. Say a car racing game is used; the prototype would only have to focus on how to control a car in order to see if it is even possible to get near a product where the user would be able to control all kinds of vehicles.

\pagebreak

\subsection{Immersive Racing Game Experience} \label{nd3}
\subsubsection*{Naïve Design}
\begin{figure}[h]
\centering
\includegraphics[width=5.5in]{NDesign3}
\caption{Concept illustration of idea 3}
\label{fig:ndesign3}
\end{figure}
The theory behind "Immersive Racing Game Experience" is to use a static webcam to track a user's movement/face by recognition and thereby allowing enhanced immersive gameplay by the use of real like movement as required in the specific computer game.


By tracking the whole body you are able to navigate in the game, since we are going to make a vehicle controller based on the user we have got many possibilities. If the focus is on car games it would be obvious to make the webcam track one's movements like someone driving a real car.


This use of movement tracking can generally be used in games that requires varied movement such as racing games, first person shooters, puzzle games, like Portal, and others, where the player acts as a human/human-like figure with movement functionalities within the game itself.

\subsubsection*{Design Analysis}
\subsubsection*{Motivation}
This idea is specifically targeted for entertaining purposes only. The theory is to enhance immersive gameplay by the use of real life movements in the specific computer game, where it is required.


The opportunity with this idea is to enhance the experience by being the subject which is used to control the in-game player in terms of movement.


\subsubsection*{Strengths and weaknesses}
Strengths:
\begin{itemize}
\item Could contribute to the already evolving technology of virtual reality.
\item Would be a cheaper alternative to more complicated ways of immersion, such as the Wii, provided that the user has enough space.
\item The user is able to get some sort of exercise while playing a game.
\item Could be a multiplayer experience.
\end{itemize}
Weaknesses:
\begin{itemize}
\item Requires a large amount of space to interact in.
\item Could limit social aspect of games.
\item People might not have the required equipment and space to make this game fully functional.
\end{itemize}


\subsubsection*{Prototype - Mock-up}
An early prototype for this concept could be a hallway with a screen and a webcam at the end. In the hallway could be obstacles that the user can hide behind as part of a game. Software will have to be made to translate the data from the camera into input to a demo game.

A prototype would clarify if users would like the idea of a racing game controlled by movements. A prototype could help show the technical limitations of this concept.


\subsubsection*{Target Groups}
The ideal target group for this solution is people who like to play games, but feel that the control options available are insufficient/uninspiring, and would like the games to be more immersive.


\subsubsection*{What do we know / what do we need to know more about}
The concept is somewhat similar to that of the Kinect as far as movement goes. The innovative part of the concept is that the player will be able to see himself on the monitor, and will be able to drive the car around by using steering wheel movements, speed up and down movements and other kind of movements that are needed to control the car or options in the game.

The Kinect is a state of the art product in this line of gaming, so it will, for obvious reasons, be natural to investigate how it works.


\subsubsection*{How much research has to be done in this area (from 1-5, where 5 is most)?}
\begin{itemize}
\item State of the Art: 4
\item Full body motion capture: 5
\item Perception: 2
\item Player motivation: 5
\end{itemize}


\subsubsection*{Investigations}
For the user to be able to control the game via the webcam, a study upon how you analyze and process an image input from the camera has to be done. The focus here will be how to find and analyze an object/person in the picture and use the person’s movements and gestures to apply some action in the game.

\pagebreak

\subsection{Pointing controller} \label{nd4}
\subsubsection*{Naïve Design}
\begin{figure}[h]
\centering
\includegraphics[width=5.5in]{NDesign4}
\caption{Concept illustration of idea 4}
\label{fig:ndesign4}
\end{figure}

The idea behind this concept is to use a webcam to track finger(s) on a plain surface to register movement and functionalities of the designated game. By relocating the finger(s), depending on the game, the input will act accordingly to the specific game/functionality.

Also, to incorporate multiplayer functionality, several players can be recognised and because of limited movement to control the specific game, a single webcam would be adequate to keep track of several players.
\bigskip

Illustration \ref{fig:ndesign4}, example 1 illustrates the finger movement, and predetermined possible movement/control functionalities of the setup.

Illustration \ref{fig:ndesign4}, example 2 shows the rotation of the players, can be adjusted down to the individual to address the angle of the screen and the angle of control that the specific player so desires.


\subsubsection*{Design Analysis}
\subsubsection*{Opportunities}
The idea is based on visual computing where there is no need for controllers. A simple webcam and a plain surface will suffice to create the specified game experience where you control the functionalities with the tip of your finger(s). This method of controlling the vehicle can be modified to combine alternative methods of controlling the game, but also customization of alternative controlling methods possible, since there are buttons to limit the possibilities.

\subsubsection*{Strength/Weakness}
The \textbf{strength} of the design is first of all the simple hardware requirements which is easy accessible and plausibly with a very simple setup. 
\bigskip

Also, by not limiting ourselves to a controller which is already developed with a limited amount of functionalities, it is possible to imitate all of the preexisting functionalities and exceed those, since there are no button limits, but rather space limit depending on the single player/multiplayer experience.
\bigskip

The \textbf{Weakness} of the design is that by utilizing a camera for tracking movements, you do not get any sort of haptic feedback under any circumstances during the experience.
\bigskip

Also, it is possible that the amount of functionalities that must be present in order to be comparable in terms of functionalities with other game platforms, is too much to cope with, since there are no visual interface lying on the surface while you are playing. Only a visual interpretation of the game mechanics that might appear on the screen.

\subsubsection*{Target groups}
The target group is people whom are interested in playing vehicle games on consoles and plausibly already does so. Preferably we are able to aim at people who are actually playing/performing vehicle gaming on a higher level, in order to research whether or not our take on a plausible alternative to preexisting methods is even a possible solution.

\subsubsection*{What to know}
\noindent\textbf{Theory/Research} - First of all we need to figure out what makes a vehicle game the preferred experience with the different controllers in order for us to establish requirements towards, what our product should be able to do and contain.
\bigskip

Furthermore, we need to establish a base of knowledge of the vehicle games in order to figure out which functionalities our product should have. Lastly, we must of course research on how to create such controller for it to be able to function to a plausible large number of games, where the controller would be equally efficient as the preexisting. 
\bigskip

\noindent\textbf{Theory/Research}\newline
\noindent\textbf{Users} - First of all we must establish criteria towards what the users think that the product would need in terms of design and functionality. Also, it would be plausible that the users are so used to “regular” controllers when playing vehicle games, so perhaps, also try a separate target group of people who only have little experience with vehicle games to establish a measurable amount of data of the user experience within both groups.
\bigskip

\noindent\textbf{SOTA} - The state of the Art within this matter is the pre-existing controllers on the market. The Kinect controller, the PS controller, Wii and the special developed wheels to emulate the driver experience.

\subsubsection*{What to investigate}
First of all we must investigate the image processing technology behind the idea. What we are able to do and how we can utilize that specific technology.

Next we must investigate the preexisting controllers to delimit ourselves to how we can make the product “comparable with other gaming platforms”. What should our product be able to do, can it be implemented, and how can we improve on the matter?

\subsubsection*{Prototype}
The prototype would most likely be an alternative to the pre-existing controllers in terms of key-functionality. We could possibly argue that we have created an alternative method of controlling vehicle games, but since it’s a different approach – there would be both advantages and disadvantages which would probably be evaluated individually by the test subjects of the evaluation.
\pagebreak

\subsection{Object-controlled Visual Processing} \label{nd5}
\subsubsection*{Naïve Design}
\begin{figure}[h]
\centering
\includegraphics[width=5.5in]{NDesign5}
\caption{Concept illustration of idea 5}
\label{fig:ndesign5}
\end{figure}

This idea revolves around a multicoloured, or shape-recognizable object that can be moved around in front of the webcam to create a motion capture-device that can be used as a logical vehicle controller e.g. a plate as seen in the image for controlling a car or boat etc.

The objects are going to have some sort of pattern on them to make them easier recognizable for the camera. Due to this pattern, it should be easy for the camera to pick up side-to-side as well as up-down movement, rotation, and forwards and backwards movement. It also makes it easy for several people to use the device at the same time, allowing for potential multiplayer games, or collaborative work.
 
The currently State-of-the-art products out there such as Kinect, PlayStation Move or Wii only allows for 2-4 players simultaneously. So maybe we will be able to create a cheap alternative to those optional products that are furthermore able to sustain more players. This idea could also enable multiplayer online, so people might be able to pick up their plate and join the game with an avatar that reacts to the object’s movement.
	
In order to make this, knowledge about image processing needs to be known. However because the pattern the camera should recognize is a pre-defined static object, it should prove less of a challenge than with a varying object such as a human hand.

\subsubsection*{Design Analysis}
\subsubsection*{Motivation}
The concept behind this idea, is replacing the traditional mouse and keyboard, with a wireless (and powerless) solution to be able to play vehicle games in an interactive, logical and intuitive fashion. 

We are motivated by the fact that products such as the Wii or Kinect seldom can sustain more that 2-4 players, and usually only come with 2 controllers, extra controllers can be purchased, but they usually cost a lot.

\subsubsection*{Strengths / weaknesses}
Strengths:
\begin{itemize}
\item Conceptually, the user would be able to use this controller from a rather great distance (though with diminished accuracy, as a side effect).
\item The physical aspect of the mouse could simply be a sticker, or something similar carrying a 	specific pattern, thus removing the need for a power source.
\item The number of players could be limited only by the restrictions of the game.
\end{itemize}
Weaknesses:
\begin{itemize}
\item The accuracy of the device is highly dependent on the skills of the user
\item Depending on the quality of the webcam, distance could very quickly become an issue for 	accuracy, as coding can only do so much.
\end{itemize}

\subsubsection*{Target groups}
The target group could be anyone who are interested in playing interactive games. 

\subsubsection*{What do you know / what do you need to know?}
For this product, a strong knowledge of image processing is the key:
\bigskip

A) Knowledge of how to capture, record and convert a specific pattern (like the ones in the above illustration) is required, in order for the program to reliably record the movement of that specific object, and not random background objects.
\bigskip

B) Knowledge of converting the data from A) into movement on the computer screen is needed. This could further be expanded upon, to allow for different settings (for different programs).

\subsubsection*{What would you be investigating?}
The main focus would be investigation regarding image capturing and processing.

Additionally we would need to gather some information regarding how web cams work, how the different types differ from one-another, how this will affect the program and how (if possible) to counteract this.
\bigskip

Networking would also be required to investigate if we were to implement an online solution.

\subsubsection*{What could be an initial prototype / mock-up?}
An initial prototype/mock-up is not easy to make, as the solution is rather simple. One solution, however, could be to create a functional program, with only limited actions available, such as simply moving the mouse around, using image capturing.

\pagebreak

\subsection{Race-making game} \label{nd6}
\subsubsection*{Naïve Design}
\begin{figure}[h]
\centering
\includegraphics[width=5.5in]{NDesign6}
\caption{Concept illustration of idea 6}
\label{fig:ndesign6}
\end{figure}

This idea is inspired by some of the racing tracks made for kids to build. You put the track together, and then you have a controller that can only control the speed of your car and nothing else. The goal is then to try and keep your car on the track, while still going at a maximum speed.
\bigskip

Of course what we should be explaining here, is the controller and not the game itself. The controller has a board (not in the above picture), where you place down the bricks as “points” where you would like the track to go. The game then connects these points with lines, and there you have the track! This would use image processing to figure out where the bricks were placed.
\bigskip

You then use your finger to control the speed of the car, trying to make it go fast, but not too fast. You need to put a coloured thing on your finger so the camera can see where it is placed.
\bigskip 

Note that this idea could be combined with some of the other ideas explained, and the control schemes could be altered to give more control of the car.

\subsubsection*{Design Analysis}
\subsubsection*{Motivation}
This idea has roots in our childhood, and because of that it is something that we can relate to.

The game itself isn’t too complicated, with the hardest part probably being how to connect the dots to form a track, and making the car controls feel nice. The image processing part of it should be fairly simple.

This idea could also potentially require one to build a table of some sort for the bricks to be placed on. This means that there will also be some work in creating something physical.

\subsubsection*{Strengths/weaknesses}
Strengths:
\begin{itemize}
\item The idea is something that can be related to.
\item Could be cheaper than buying an actual racing track.
\item Allows users to be creative by being able to create their own things with the provided material.
\item Allows for social interaction.
\end{itemize}
Weaknesses:
\begin{itemize}
\item Could potentially require two cameras, however this can be fixed by using the board where you place the bricks as the “control-space” as well.
\end{itemize}

\subsubsection*{Target group}
The target group would be anyone who plays games. Younger people might be a greater target group however, since they might’ve played the physical version of the game, and feele nostalgia when it comes to the game.
\bigskip

The game should be really simple and intuitive to use, so you don’t need a lot of technological expertise to use the product.

\subsubsection*{Knowledge}
We know for a fact that some students have made tables that are able to detect objects on top of it before, so some research has already been done in this area, and we’ll have some other products to compare ours to. The same has been done with color recognition.
\bigskip

In order to create this idea, we would need knowledge about image processing to create the controlling features. We would also need knowledge about how to create the actual game. This could be done in a lot of programs such as Unity, Flash, Processing and more. We could potentially also create it from scratch using c++, however this would most likely not be preferred.
\bigskip

We also need knowledge about vectors and vector calculation in order to create the track, and some knowledge about 2d physics, to make the car act in a way we want. However a program such as Unity already has a physics engine, potentially saving us time.
\bigskip

We can make loads of prototypes of this idea, and it has a lot of “steps” on which we can test the game. We can test the initial colour recognition without the need of anything else, we can test the creation of the track without the need of anything else, and we can test the car ‘mechanics’ in a way without the need of much, except a track to follow.

\pagebreak

\subsection{Tabletize your computer!} \label{nd7}
\subsubsection*{Naïve Design}
The idea for this design is, using a standard webcam, simulating the rotation measuring devices of modern tablets and smart phones, allowing computer users the same intuitive means of controlling games.
\bigskip

As the webcam is recording, edges will be recognized (using image processing), and the movement of said edges will be converted into movement on the computer. In other words, movement within the captured pictures will be replacing the tilting sensors of the tablets and smart phones.

\subsubsection*{Design Analysis}
\subsubsection*{Motivation}
The motivation behind this concept, is to create software, that will allow a webcam to replicate the tilting and rotation sensors ordinarily found in e.g. tablets and smartphones. This would allow desktop and laptop computer users to immerse themselves in similar games, i.e. games where the controls are more intuitive than simply using something like the WASD keyboard keys.

\subsubsection*{Strengths and weaknesses}
Strengths:
\begin{itemize}
\item Allows for a more immersive experience
\item Expands the range of games available to computer users even further.
\item Could possibly be utilized for other purposes, where a tilt sensor could be useful.
\end{itemize}
Weaknesses:
\begin{itemize}
\item Hardware not built for this kind of action, unless using an external webcam.
\item Potentially hard / impossible to do general calibration, causing the device to needing 	recalibration after either every  game session or after every time it is moved from location to 	location.
\item Unless using external webcam, the setup can be both heavy and cumbersome, making the 	device less intuitive than the alternative, thus defeating the point.
\end{itemize}

\subsubsection*{Investigation}
What do we know / what do we need to know more about:

We will most likely be using openFrameworks (C++ coding) to process the images into movement data. What we need to know is, how to convert the stream of images from the webcam into usable movement data, ie. how to detect the movement in the images reliably. Furthermore we need to convert this data into actual movement in a game.

We will need to investigate similar, professional, solutions, sort of like the tablets and smartphones, but we should try to see if we can find someone who have done the same without the use of tilt-sensors.
\bigskip

We will need to research a lot of image processing, and figure out how we're going to approach and process the movement, ie. recorded movement -> data -> output movement, and how much physical movement should account for how much movement on screen. 

\subsubsection*{Initial prototype}
An initial prototype is not easy to make, as it is highly software based, and thus will more or less be finished, once an initial prototype is ready for testing. "Simulations", or Wizard of Oz-experimentations, could be used, as this would give some idea of how the program will work and feel for the users, while being very time efficient.



\clearpage
\section{Testing-Time measurements} \label{app:time}
Acquired lap completion data.

\begin{table}[!htbp]
\centering
\begin{tabular}{| l | c | c | c |}
\hline
\textbf{Subject no.} & \textbf{Controller sequence} & \textbf{Product time: min/sec} & \textbf{Sota time:}\\
\hline
\# 1 & PRODUCT $ \rightarrow $ SOTA & 3:28 & 3:11\\
\# 2 & SOTA $ \rightarrow $ PRODUCT & 4:58 & 3:13\\
\# 3 & PRODUCT $ \rightarrow $ SOTA & 5:14 & 3:05\\
\# 4 & SOTA $ \rightarrow $ PRODUCT & 4:49 & 3:04\\
\# 5 & PRODUCT $ \rightarrow $ SOTA & 4:46 & 3:07\\
\# 6 & SOTA $ \rightarrow $ PRODUCT & 3:29 & 3:09\\
\# 7 & PRODUCT $ \rightarrow $ SOTA & 4:03 & 3:11\\
\# 8 & SOTA $ \rightarrow $ PRODUCT & 3:59 & 3:03\\
\# 9 & PRODUCT $ \rightarrow $ SOTA & 3:45 & 3:05\\
\# 10 & SOTA $ \rightarrow $ PRODUCT & 4:29 & 3:17\\
\# 11 & PRODUCT $ \rightarrow $ SOTA & 3:37 & 3:11\\
\# 12 & SOTA $ \rightarrow $ PRODUCT & 4:54 & 3:21\\
\hline
\end{tabular}
\caption{Description of control sequence and recorded lap records} \label{tab:timemeasurements}
\end{table}

\noindent\textbf{Subject no:} Specifies the number of the participant.


\noindent\textbf{Controller sequence:} Indicates the sequence for which the contestant played a lap with either of the controllers first and then afterwards the other.


\noindent\textbf{Product time:} Describes the time for which the lap has been completed with the developed product.


\noindent\textbf{SOTA time:} describes the time for which the lap has been completed with the Xbox controller (State of the Art controller).
\clearpage
\section{Testing-Observation notes} \label{app:obs}
The left brackets cover the observations notes for the laps where the test subjects used the constructed controller.

The right bracket covers the noted observations, where the Xbox controller (state of the art) was used.

\begin{table}[!htbp]
\centering
\begin{tabular}{| p{3.4in} | p{2in} |}
\hline
	\textbf{Test Subject \# 1: Product} & \textbf{SOTA}\\
\hline
	\begin{itemize}
		\item Slight function delay
		\item Breaking was non responsive. Wheel was not forward enough to enable it.
		\item The wheel was placed very far forward, in order to accelerate.
		\item Random activation of the camera view change. Occurred +5 times.
		\item Over exaggerated the turning motion to turn.
		\item After practice, enabling the “change view” was improved and functional
		\item Change view happens for no reason.
		\item Camera change view is enabled by flipping the wheel over the shoulder.
		\item Subject quote: “shoulder hurts”
		\item Subject rarely breaks.
		\item No in-game crashes.
	\end{itemize}
	&
	\begin{itemize}
		\item Easier to control
		\item Faster function response
	\end{itemize}
	\\
\hline
\end{tabular}
\end{table}


\begin{table}[!htbp]
\centering
\begin{tabular}{| p{3.4in} | p{2in} |}
\hline
	\textbf{Test Subject \# 2: Product} & \textbf{SOTA}\\
\hline
	\begin{itemize}
		\item No noticeable braking during race
		\item Change camera is enabled several times on purpose during game
		\item Game was reset because of technical difficulties – not associated with product.
		\item Wheel (board) was tilted, excessively when car was out of control (things were going bad).
		\item Several occasions: Car was wobbling. Hard left, or hard right.
		\item Arms of subject are placed very low. Near thighs in the sitting position.
		\item Noticeable frustration of the test subject.
		\item Slight function delay
		\item Tried to enable “change view”. Nothing happened
		\item Lost control of the forward/backward mechanics  - Trouble activating them after the other.
		\item Kept arms straight, (wheel close to screen) while driving.
		\item Random change of view occurred 3-4 times.
		\item After in game crash, subject had difficulties getting control over the vehicle, getting back on track. Forward/backward \& turning.
		\item Change of view was activated once on purpose.
	\end{itemize}
	&
	\begin{itemize}
		\item Several occasions: Car was wobbling. Hard left, or hard right.
		\item In game crash
		\item Over eager
	\end{itemize}
	\\
\hline
\end{tabular}
\end{table}


\begin{table}[!htbp]
\centering
\begin{tabular}{| p{3.4in} | p{2in} |}
\hline
	\textbf{Test Subject \# 3: Product} & \textbf{SOTA}\\
\hline
	\begin{itemize}
		\item Subject was driving off the track repeatedly.
		\item Arms are held very low.
		\item Subject is wearing a red shirt and might interfere with the color registration.
		\item Incidents of unwillingly turning right.
		\item Camera view was changed twice.
		\item Uncontrollable steering during most of the lap.
		\item Subject quote: “ I would have rage quitted long ago”.
		\item Slight delay spikes noted.
		\item Could only turn a lot. Small turns were over exaggerated and therefore drove off the track.
		\item Activated the change view function twice.
		\item Function activations were unstable. Speed/ braking.
		\item After in game crash, the subject was struggling with enabling reverse and turn to get back on track. Non responsive.
		\item Camera view changed randomly, 3-4 times.
		\item Kept the controller in straight arms – aiming it close towards the computer.
		\item Had to pull the wheel all the way back in order to brake. Close to the body
		\item When turning, only big turns were possible. Always too much left or right.
		\item Note: subject wearing red shirt. Might cause complications.
	\end{itemize}
	&
	\begin{itemize}
		\item Subject is able to focus on the in-game driving line, indicating control of the vehicle.
		\item No in game crashes.
		\item Good run, with no noted difficulties.
		\item Subject note: “more precise”
	\end{itemize}
	\\
\hline
\end{tabular}
\end{table}


\begin{table}[!htbp]
\centering
\begin{tabular}{| p{3.4in} | p{2in} |}
\hline
	\textbf{Test Subject \# 4: Product} & \textbf{SOTA}\\
\hline
	\begin{itemize}
		\item In game crash noted.
		\item Overturning a lot therefore ends up off the track a lot.
		\item Wheel board is tilted when subject is experiencing some unexpected behavior of the vehicle. ex. Over turning.
		\item Vehicle is occasionally turning right.
		\item Subject’s arms are held low, close to thighs sitting down and in straight arms – close to the camera.
		\item Drop in frame rate during lap for several seconds.
		\item Change camera view are unwillingly activated
		\item Drop in frame rate for about 20 seconds.
		\item Game was restarted. The gear could not get above 1st. Settings was sat to manual.
		\item Over steering noted.
		\item Problems with correctly placing the controller. Referring to the placement in space, (point of forward/backward etc.).
		\item Random activation of change view. 6-7 incidents.
		\item When subject tries to reverse – view is changed.
		\item Turning is only exaggerated, making handling of the vehicle difficult.
	\end{itemize}
	&
	\begin{itemize}
		\item One in game crash noted.
		\item Subject is able to focus on the in-game driving line, indicating control of the vehicle.
		\item Better handling is commented.
	\end{itemize}
	\\
\hline
\end{tabular}
\end{table}


\begin{table}[!htbp]
\centering
\begin{tabular}{| p{3.4in} | p{2in} |}
\hline
	\textbf{Test Subject \# 5: Product} & \textbf{SOTA}\\
\hline
	\begin{itemize}
		\item Slight function delay
		\item Over turning, small turns were hard.
		\item Sudden stop in acceleration while driving.
		\item Change camera view is randomly activated. 4 times noted.
		\item Arms are held low, close to thighs
		\item In game crash noted
		\item Change camera function is randomly enabled.
		\item Driving off track because of large turns.
		\item Reverse was difficult. Hard to enable.
	\end{itemize}
	&
	\begin{itemize}
		\item Subject is able to focus on the in-game driving line. Indicating control of the vehicle.
		\item Controller is resting in lap.
		\item In game crash noted.
	\end{itemize}
	\\
\hline
\end{tabular}
\end{table}


\begin{table}[!htbp]
\centering
\begin{tabular}{| p{3.4in} | p{2in} |}
\hline
	\textbf{Test Subject \# 6: Product} & \textbf{SOTA}\\
\hline
	\begin{itemize}
		\item Function delay – Turning.
		\item Small notions (movement) did not do anything.
		\item Random clusters of not accelerating, happens randomly.
		\item Subject did not use camera view option.
		\item Subject holds the wheel in straight arms towards/close to the camera.
		\item In game crash noted.
	\end{itemize}
	&
	\begin{itemize}
		\item Follows the in game line – indicating control of the vehicle.
		\item Better control of the “small” turns.
		\item Can feel the enabled driving aid function.
		\item Subject quote:” the acceleration is weird”.
	\end{itemize}
	\\
\hline
\end{tabular}
\end{table}


\begin{table}[!htbp]
\centering
\begin{tabular}{| p{3.4in} | p{2in} |}
\hline
	\textbf{Test Subject \# 7: Product} & \textbf{SOTA}\\
\hline
	\begin{itemize}
		\item Wheel is held in straight arms.
		\item Lots of big turns, ending up off track.
		\item Random activation of camera change. Occurred once.
		\item Moving the wheel to the left when trying to turn left instead of making the left rotation notion.
		\item Drop in frame rate for seconds.
		\item Only big turns is happening. No small turns difficult.
		\item Controller moved right – None responding activation of change camera view.
		\item Controller is moved forward – braking. Noted: Because of the auto steering.
		\item Randomly enabled change of view twice.
		\item Drop in frame rate is observed.
	\end{itemize}
	&
	\begin{itemize}
		\item More sensitive, in comparison.
		\item Holds the in-game line.
		\item No in game crash.
	\end{itemize}
	\\
\hline
\end{tabular}
\end{table}


\begin{table}[!htbp]
\centering
\begin{tabular}{| p{3.4in} | p{2in} |}
\hline
	\textbf{Test Subject \# 8: Product} & \textbf{SOTA}\\
\hline
	\begin{itemize}
		\item Subject note: Very “binary” control.
		\item Subject note: requires adaption to control.
		\item Subject is keeping arms straight, close to the screen.
		\item Noted in game crash.
		\item No activation of camera change function.
		\item Only “big” turns are possible. Small turns are left to the auto steering.
		\item Ignoring small turns, since turning would result in over steering. ( commented while playing).
		\item Tried to brake. Nothing happened. (subject did not move backward enough).
	\end{itemize}
	&
	\begin{itemize}
		\item Subject quote: Had time to look at the in game map while driving.
	\end{itemize}
	\\
\hline
\end{tabular}
\end{table}


\begin{table}[!htbp]
\centering
\begin{tabular}{| p{3.4in} | p{2in} |}
\hline
	\textbf{Test Subject \# 9: Product} & \textbf{SOTA}\\
\hline
	\begin{itemize}
		\item Decrease in frame rate. About 10 sec.
		\item Turning was exaggerated.
		\item Successfully enabled camera view twice
		\item Subject only tried turning during “big” turns.
		\item Left small turns to the auto steering.
		\item Subject is tilting the wheel while turning.(only rotation is required)
		\item Arms are held straight, towards the camera.
		\item Arms are held high
		\item No random activation of camera is occurring.
		\item Decrease in frame rate is noted.
	\end{itemize}
	&
	\begin{itemize}
		\item Subject is eager.
		\item In game crash.
	\end{itemize}
	\\
\hline
\end{tabular}
\end{table}


\begin{table}[!htbp]
\centering
\begin{tabular}{| p{3.4in} | p{2in} |}
\hline
	\textbf{Test Subject \# 10: Product} & \textbf{SOTA}\\
\hline
	\begin{itemize}
		\item Drop in frame rate.
		\item Controller only reacts to big turns.
		\item Random activation of camera change. Occurred +12 times.
		\item Activation of camera intentional happened twice.
		\item Arms are held low.
		\item Subject quote:” likes the authenticity of turning”
		\item Over steering a lot.
		\item Drop in frame rate.
		\item Random activation of camera view.
		\item Arms are held very low.
	\end{itemize}
	&
	\begin{itemize}
		\item In game crash.
		\item Falling of the track occasionally.
		\item Eager.
		\item Subject note: Difficult to control games with the sticks. Referring to the SOTA controller
		\item Easier to control.
		\item Subject note: the design of the controller is better
	\end{itemize}
	\\
\hline
\end{tabular}
\end{table}


\begin{table}[!htbp]
\centering
\begin{tabular}{| p{3.4in} | p{2in} |}
\hline
	\textbf{Test Subject \# 11: Product} & \textbf{SOTA}\\
\hline
	\begin{itemize}
		\item Over turning. Losing control over the vehicle
		\item Random activation of change view. 2-3 times.
		\item Moves forward/backward, a little, indicating that the subject expects to accelerate more or less depending on how close the subject is.
		\item Wobbles a lot.
		\item Random activation of change view.
		\item Arms are straight.
	\end{itemize}
	&

	\\
\hline
\end{tabular}
\end{table}


\begin{table}[!htbp]
\centering
\begin{tabular}{| p{3.4in} | p{2in} |}
\hline
	\textbf{Test Subject \# 12: Product} & \textbf{SOTA}\\
\hline
	\begin{itemize}
		\item Subject is flipping the wheel (holding the top side towards the camera), to steer. Resulting in several errors.
		\item Drop in frame rate.
		\item Accidental activation of view change. (subject flips the wheel, pointing downward). Occurs 7+ times.
		\item Moves the wheel a lot to left/right when turning, therefore change view is enabled.
		\item Frame rate reduction. About 5 sec.
		\item Over turning noted.
	\end{itemize}
	&

	\\
\hline
\end{tabular}
\end{table}
\clearpage
\section{Interview notes} \label{app:interview}
The interview is recorded and de-constructed piece by piece. Converted into verbatim transcriptions for which, each point of the conversation is denoted as list of arrays of sentences.

These sentences do not represent one single test subject's opinion, but is a reconstruction of all sentences which has been provided as response from all test subjects.

\subsection{Product responses}
\begin{itemize}
\item[]
\begin{table}[!htb]
\centering
\begin{tabular}{| p{5.5in} |}
\hline
	\cellcolor{NotSkyBlue}\textbf{Did you experience any Function delay that affected the gameplay?}
	\\
	\hline
	\begin{itemize}
        \item Experienced varied function delays which was annoying.
        \item Yes. Mainly when  I changed view.
        \item Yes. Continuously delays. Typical a second or more.
        \item When you turned, there was a short delay. More when you break. To get it to break, it took a long time.
        \item Yes, there was a delay. Not as much a delay, but more as hard to register the small movements/notions.
        \item Yes. Either it turns a lot, or no turning at all. It got easier the more I tried.
        \item Delays when you turn left or right. Like having to make a very hard gesture for anything to happen.       
        \item You had to wait a little bit to see if a function worked or not.
        \item Yea. Turning. I don’t know about the speeder. I just tried to put the pedal to the metal.
        \item Turning delay was about a second.
        \item It laggs a little. But you got used to it.
        \item No, but I accidentally activated the screen shift. It’s about finding the right method for it.
        \item No. I didn’t think about that. The only problem was that it kept changing screen view point.
	\end{itemize}
	\\
	\hline
\end{tabular}
\end{table}

\item[]
\begin{table}[!htb]
\centering
\begin{tabular}{| p{5.5in} |}
\hline
	\cellcolor{NotSkyBlue}\textbf{Did you experience any Non responsiveness of functions that affected the gameplay?}
	\\
	\hline
	\begin{itemize}
        \item Not so responsive when turning the wheel. Sometimes it was really fast as well.
        \item Indicated that there were more problems with the functions activating itself, rather than non-responding functions. Ex. Change of view.
        \item The view or reverse did not work several times
        \item No. I did not notice it not turning. But changing camera enabled randomly.
        \item I had problems concerning the acceleration. I did not know if I was accelerating or not. (maybe because of the lighting or sound).
        \item Acceleration and braking worked pretty good.
	\end{itemize}
	\\
	\hline
\end{tabular}
\end{table}

\item[]
\begin{table}[!htb]
\centering
\begin{tabular}{| p{5.5in} |}
\hline
	\cellcolor{NotSkyBlue}\textbf{Did you experience any difficulty activating game functions that affected the gameplay?}
	\\
	\hline
	\begin{itemize}
        \item Yes. Especially when trying to accelerate.” – Indicated moving the controller forward.
        \item Yes.  There was times where the control stopped working, or didn’t activate at all.
        \item Yes. Moving the controller too far behind was difficult.
        \item The activations were fine. Could require adjustments.
        \item Changed camera as an accident. Afterwards I didn’t dare try switching it back
        \item Very hard in the beginning. Once I got used to the distances, it got a lot easier.
        \item While driving reverse, it was too far behind. Like the distance from accelerating to de accelerating was too big.  It should be displaced a little.
        \item The camera view. I could not get it to work. It randomly changed back. I didn’t understand that.
        \item It worked pretty good except the sensitivity of the view point. Perhaps some body motions which conflicts with the  program.
        \item No, I won’t say that.
        \item It was mostly with the speed up. To figure out if it actually speeded up. Found that I had to flip the controller a little forward. But the camera should be a little further behind for it to work.

	\end{itemize}
	\\
	\hline
\end{tabular}
\end{table}

\item[]
\begin{table}[!htb]
\centering
\begin{tabular}{| p{5.5in} |}
\hline
	\cellcolor{NotSkyBlue}\textbf{Did you experience any complications with enabling consecutive game functions (using functions together in context) that affected the gameplay? }
	\\
	\hline
	\begin{itemize}
        \item Sometimes it changed camera view all by itself. Changing it back worked quite well.
        \item No
        \item Because of the delay, suddenly changing direction became difficult.
        \item Acceleration and turning:  could go a little wrong.  When accelerating it was a little too much and turning, then the car is drifting off track.
        \item It’s hard since you must always keep your arms straight.  A tipping function to stopping or accelerating would be better. Then your arms is a little closer and therefore not so hard.
        \item When I wanted to voluntarily change view point, then it worked.
        \item That could be, when you dragged a little behind when trying to turn.
        \item I didn’t try to enable the screen view. But it randomly activated while I tried to turn.
        \item It was pretty difficult to start with, but that was because the car ended on the side, so there was not a lot of track to turn on. Perhaps it had a bit of difficulty registering what I wanted it to do.

	\end{itemize}
	\\
	\hline
\end{tabular}
\end{table}

\item[]
\begin{table}[!htb]
\centering
\begin{tabular}{| p{5.5in} |}
\hline
	\cellcolor{NotSkyBlue}\textbf{Did you experience any complications with the controller movement and handling that affected the gameplay?}
	\\
	\hline
	\begin{itemize}
        \item Indicates putting the controller forward, overextending body is plausibly a complication.
        \item You get very tired of holding the controller, therefore you move it further and further down resulting in the controller suddenly not working as intended.
        \item The handling of the controller could have been constructed a little better. (also arms got tired)
        \item If the controller were more responsive, I think the handling would be okay.
        \item Once I found out how to handle the controller it was pretty funny.
        \item Not so precise
        \item Not normal
        \item Would like to “flip the controller” instead of turning with the arms.
        \item The controller was funny, but more “Wii like”. The controller is funny but there are many others that I would rather use.
        \item Arms hurt, but else quite funny.
        \item I would rather have a button for acceleration, so I could relax a little more.
        \item The only problem, was that it changed the view point. Also I was missing some sound.
        \item A little more weigh would have been good.
        \item Clumsy handling.
        \item I figured it out. I also found out, that if I placed the controller forward, I speeded up more.
        \item Feels too light. If it was on a stand, it would have been easier.
        \item Difficult. But you got used to it. So it can be done if you have strong arms.
        \item I think it was pretty cool. But the conceptual change of view point was disturbing. 
        \item It was difficult. You easily turned a lot one way. And then turning the other way.

	\end{itemize}
	\\
	\hline
\end{tabular}
\end{table}

\item[]
\begin{table}[!htb]
\centering
\begin{tabular}{| p{5.5in} |}
\hline
	\cellcolor{NotSkyBlue}\textbf{Misc. notes. (comments that do not fall under any of the mentioned questions.)}
	\\
	\hline
	\begin{itemize}
        \item Could not see the purpose of the function which changed the camera view.
	\end{itemize}
	\\
	\hline
\end{tabular}
\end{table}

\end{itemize}

\clearpage

\subsection{Sota controller responses}
\begin{itemize}

\item[]
\begin{table}[!htb]
\centering
\begin{tabular}{| p{5.5in} |}
\hline
	\cellcolor{NotSkyBlue}\textbf{Did you experience any Function delay that affected the gameplay?}
	\\
	\hline
	\begin{itemize}
        \item No
        \item No
        \item No
        \item Not something I noticed. Perhaps there was some, but it could be the game that tried to correct.
        \item No
        \item No
        \item Very sensitive
        \item No
        \item Small problem. I hold my finger on the stick that I am not using. It annoys me. Else no problems.
        \item No, it was easier to handle
        \item No
        \item Not at all
	\end{itemize}
	\\
	\hline
\end{tabular}
\end{table}

\item[]
\begin{table}[!htb]
\centering
\begin{tabular}{| p{5.5in} |}
\hline
	\cellcolor{NotSkyBlue}\textbf{Did you experience any Non responsiveness of functions that affected the gameplay?}
	\\
	\hline
	\begin{itemize}
        \item No
        \item I held accelerate in all the time and I don’t think that it accelerated.
        \item No
        \item No
        \item Easy to use. You could make small turns. It was easy to make the small movements to turn
        \item No
        \item No
        \item No
        \item No
        \item No, but I’ve always had trouble with joysticks. So yours is more intuitive.
        \item No
        \item No

	\end{itemize}
	\\
	\hline
\end{tabular}
\end{table}

\item[]
\begin{table}[!htb]
\centering
\begin{tabular}{| p{5.5in} |}
\hline
	\cellcolor{NotSkyBlue}\textbf{Did you experience any difficulty activating game functions that affected the gameplay?}
	\\
	\hline
	\begin{itemize}
        \item Worked quite well. Except I couldn’t find out how to change the camera view. It actually worked out better with your controller.
        \item No
        \item No
        \item No, not at all
        \item No, not really. If you just place your fingers correct you can also change camera angle.
        \item No
        \item No, I won’t say
        \item No, I don’t think so. Except the automatic braking. It did that a lot
        \item No
        \item Not anything else, than concentrating more in remembering where the specific controls are.
        \item No
        \item No

	\end{itemize}
	\\
	\hline
\end{tabular}
\end{table}

\item[]
\begin{table}[!htb]
\centering
\begin{tabular}{| p{5.5in} |}
\hline
	\cellcolor{NotSkyBlue}\textbf{Did you experience any complications with enabling consecutive game functions (using functions together in context) that affected the gameplay?}
	\\
	\hline
	\begin{itemize}
        \item No
        \item No
        \item No
        \item No problems at all.
        \item No, it was pretty easy.
        \item No
        \item No
        \item No
        \item No
        \item No I don’t think so. Though there was some automatic braking.
        \item No

	\end{itemize}
	\\
	\hline
\end{tabular}
\end{table}

\item[]
\begin{table}[!htb]
\centering
\begin{tabular}{| p{5.5in} |}
\hline
	\cellcolor{NotSkyBlue}\textbf{Did you experience any complications with the controller movement and handling that affected the gameplay?}
	\\
	\hline
	\begin{itemize}
        \item No. I like the fact that you can put the controller on your legs. That was way better.
        \item It’s an Xbox controller, so one of those of course I don’t like (joking)
        \item More precise
        \item No, it was good to handle.
        \item Pretty comfortable.
        \item What work better is the small movements, since you are adjusting all the time. It works better
        \item Definitely the biggest difference is the small notions versus the big notions.
        \item The steering pen is pretty annoying. Else it is very comfortable
        \item No, pretty nice.
        \item No
        \item No, it is easier to handle. But the color controller is more authentic. Design wise is the most memorable.
        \item It was easy to handle
        \item More used to this controller. It’s more intuitive. Where your controller is more like  a steering wheel. This controller was a little more precise.

	\end{itemize}
	\\
	\hline
\end{tabular}
\end{table}

\end{itemize}
\clearpage
\section{Concept development and concept into categories} \label{app:concept}
From the categories which have been defined, broader groups of similar concept are identified to form categories of data which is approximately similar in terms of context. 
These categories will be deemed to be of similar content and will include a detailed interpretation of the specific categories.
\bigskip

\noindent\colorbox{NotLavender}{\textbf{Delay activating functions}}\label{eval:delay}\newline
These groups of responds are deemed to be of similar content and describe observations that indicate that either the observer is observing delay or the test subject is expressing it, from where a certain action is activated till the actual desired action is performed.

\begin{itemize}
\item Slight function delay.
\item Slight function delay.
\item Slight delay spikes noted.
\item Slight function delay.
\item Function delay – turning.
\end{itemize} 
\bigskip

\noindent\colorbox{NotGreenYellow}{\textbf{Acceleration, Reverse/braking relevance}}\newline
The following data elements describe issues and difficulties which are similar to the general control of the controller. The relationship between these notations is inability to activate a certain action, loss of control of a certain action or in any way irregular incidents which occur with the above mentioned category.

\noindent\textbf{Controller placement issues}
\begin{itemize}
\item Breaking was non responsive. Wheel was not forward enough to enable it.
\item Problems with correctly placing the controller. Referring to the placement in space, (point of forward/backward etc.).
\item The wheel was placed very far forward, in order to accelerate.
\item No noticeable braking during race
\item Lost control of the forward/backward mechanics - Trouble activating them after the other.
\item Function activations were unstable. Speed/ braking.
\item Sudden stop in acceleration while driving.
\item Random clusters of not accelerating, happens randomly.
\item Tried to brake. Nothing happened. (Subject did not move backward enough).
\item Moves forward/backward, a little, indicating that the subject expects to accelerate more or less depending on how close the subject is.
\item After in game crash, the subject was struggling with enabling reverse and turn to get back on track. Non responsive.

\end{itemize}
\noindent\textbf{Game setting relevance}
\begin{itemize}
\item Controller is moved forward – braking. Noted: Because of the auto steering.
\end{itemize}
\bigskip

\noindent\colorbox{NotSkyBlue}{\textbf{Change camera view relevance}}\newline
The change camera action accounts for the following sub categories and the content of the data is deemed to be of similar content as the mentioned sub categories.
Unintentional activation of the action({\color{Red}3.1}) account for the cases where observations which describes unwilling activation of the said control.({\color{Red}3.2}) specifies the occasions for which successful criteria has been deemed as activating the control has been listed as a category. Lastly, ({\color{Red}3.3}) specifies the data for which the test subject did not try to use the action, or was unsuccessful in trying to.

\noindent\textbf{3.1 Unintentional/accidental/unaccounted activation of the change camera setting}
\begin{itemize}
\item Random activation of the camera view change. Occurred +5 times.
\item Change view happens for no reason.
\item Random change of view occurred 3-4 times.
\item Camera view changed randomly, 3-4 times.
\item Random activation of change view. 6-7 incidents.
\item When subject tries to reverse – view is changed
\item Change camera view is randomly activated. 4 times noted.
\item Change camera function is randomly enabled.
\item Random activation of camera change. Occurred once.
\item Random activation of camera change. Occurred +12 times.
\item Random activation of camera view.
\item Random activation of change view. 2-3 times.
\item Random activation of change view.
\item Accidental activation of view change. (Subject flips the wheel, pointing downward). Occurs 7+ times.
\item Moves the wheel a lot to left/right when turning, therefore change view is enabled.
\item Randomly enabled change of view twice.
\item Controller moved right – None responding activation of change camera view.
\end{itemize}

\noindent\textbf{3.2 Active and intentional activation of the change camera setting}
\begin{itemize}
\item After practice, enabling the “change view” was improved and functional
\item Camera change view is enabled by flipping the wheel over the shoulder.
\item Change camera is enabled several times on purpose during game
\item Change of view was activated once on purpose
\item Camera view was changed twice.
\item Activated the change view function twice.
\item Successfully enabled camera view twice
\item No random activation of camera is occurring.
\item Activation of camera intentional happened twice.
\end{itemize}

\noindent\textbf{3.3 No activation of the change camera setting. Either optional or unsuccessful.}
\begin{itemize}
\item No activation of camera change function
\item Subject did not use camera view option.
\item Tried to enable “change view”. Nothing happened
\end{itemize}
\bigskip

\noindent\colorbox{NotOrange}{\textbf{Turning relevance}}\newline
Turning relevance covers the general steering left/right in game. Additionally, in-game accidents, test subject movements that affected the turning and general comments/actions which relate to any elements that explains turning. ({\color{Red}4.1})In-game steering difficulties: Overturning, covers the data for which difficulties with steering has been observed and noted. ({\color{Red}4.2})  Controller mechanics: Occasional unwilling turning specifies the observer notes for which unwilling turning was happening.

\noindent\textbf{4.1 In-game steering difficulties: Overturning}
\begin{itemize}
\item Several occasions: Car was wobbling. Hard left, or hard right.
\item Subject was driving off the track repeatedly.
\item Uncontrollable steering during most of the lap.
\item Could only turn a lot. Small turns were over exaggerated and therefore drove off the track.
\item When turning, only big turns were possible. Always too much left or right.
\item Overturning a lot therefore ends up off the track a lot.
\item Turning is only exaggerated, making handling of the vehicle difficult.
\item Over turning, small turns were hard.
\item Driving off track because of large turns.
\item Lots of big turns, ending up off track.
\item Only big turns is happening. No small turns difficult.
\item Only “big” turns are possible. Small turns are left to the auto steering.
\item Over steering noted.
\item Small notions (movement) did not do anything.
\item Ignoring small turns, since turning would result in over steering. (Commented while playing).
\item Turning was exaggerated.
\item Over exaggerated the turning motion to turn.
\item Subject only tried turning during “big” turns
\item Left small turns to the auto steering.
\item Controller only reacts to big turns.
\item Over steering a lot.
\item Over turning. Losing control over the vehicle
\item Wobbles a lot.
\item Over turning noted.
\end{itemize}


\noindent\textbf{4.2 Controller mechanics: Occasional unwilling turning }
\begin{itemize}
\item Incidents of unwillingly turning right.
\item Vehicle is occasionally turning right.
\end{itemize}


\noindent\colorbox{NotRed}{\textbf{Subject conditions, comments \& actions}}\newline
The subject conditions, comments \& actions covers the data for which observations towards postures, positioning, actions of the test subjects or conditions have noticeable relations. 

\noindent\textbf{5.1 Observed postures of the test subjects during gameplay.}
\begin{itemize}
\item Arms of subject are placed very low. Near thighs in the sitting position.
\item Kept arms straight, (wheel close to screen) while driving.
\item Arms are held very low.
\item Kept the controller in straight arms – aiming it close towards the computer.
\item Arms are held low, close to thighs
\item Subject’s arms are held low, close to thighs sitting down and in straight arms – close to the camera.
\item Subject holds the wheel in straight arms towards/close to the camera.
\item Wheel is held in straight arms.
\item Subject is keeping arms straight, close to the screen.

\end{itemize}


\noindent\textbf{5.2 Subject comments during gameplay}
\begin{itemize}
\item Subject quote: “shoulder hurts”
\item Subject note: Very “binary” control.
\item Subject note: requires adaption to control.
\item Subject quote: “I would have rage quitted long ago”.

\end{itemize}


\noindent\textbf{5.3 In-game steering recovery observations}
\begin{itemize}
\item Wheel (board) was tilted, excessively when car was out of control (things were going bad).
\item Noticeable frustration of the test subject.
\item After in game crash, subject had difficulties getting control over the vehicle, getting back on track. Forward/backward \& turning.
\item Wheel board is tilted when subject is experiencing some unexpected behavior of the vehicle. ex. Over turning.

\end{itemize}


\noindent\textbf{5.4 Observed plausible clothing difficulties}
\begin{itemize}
\item Subject is wearing a red shirt and might interfere with the color registration.
\item Note: subject wearing red shirt. Might cause complications.

\end{itemize}



\noindent\colorbox{NotGreen}{\textbf{6.	Game \& in-game crashes}}\newline
Game \& in-game crashes covers the crashes in-game, which is defined as driving off the road, losing control over the vehicle. Game crashes indicates difficulties with the game itself, being the developed software or the actual game which suffered temporarily conditions, deemed defined as a crash.

\begin{itemize}
\item No in-game crashes.
\item Game was reset because of technical difficulties – not associated with product.
\item In game crash noted.
\item Game was restarted. The gear could not get above 1st. Settings was sat to manual.
\item In game crash noted
\item In game crash noted.
\item Noted in game crash.

\end{itemize}



\noindent\colorbox{NotPurple}{\textbf{Frame rate relevance}}\newline
Frame rate relevance covers the incidents for which a significant drop in frame rate during the laps was observed.

\begin{itemize}
\item Drop in frame rate during lap for several seconds.
\item Drop in frame rate for about 20 seconds.
\item Drop in frame rate is observed.
\item Decrease in frame rate. About 10 sec.
\item Decrease in frame rate is noted.
\item Drop in frame rate.
\item Drop in frame rate.
\item Frame rate reduction. About 5 sec.
\item Drop in frame rate for seconds.

\end{itemize}
\clearpage